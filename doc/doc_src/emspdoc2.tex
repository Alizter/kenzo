\chapter {Eilenberg-Moore spectral sequence II}

\section{Introduction}

This chapter is devoted to the effective homology version of the spectral
sequence of Eilenberg-Moore, in the particular case of classifying spaces.
More precisely, let ${\cal G}$ be a simplicial group; then its classifying space 
$\bar{\cal W}{\cal G}$ is a simplicial set canonically defined\footnote
{{\bf J. Peter May}. {\em Simplicial objects in algebraic topology}, Van Nostrand, 1967.}.
Furthermore, if ${\cal G}$ is an Abelian simplicial group, then  $\bar{\cal W}{\cal G}$
is again an Abelian simplicial group with natural structure, so that the $\bar{\cal W}$
construction can be iterated. The $\bar{\cal W}$ construction is implemented in {\tt Kenzo}
only if ${\cal G}$ is reduced (${\cal G}_0$ has only one element). In particular if
${\cal G}$ is a simplicial group with effective homology, then {\tt Kenzo} constructs
a version of $\bar{\cal W}{\cal G}$ also with effective homology. 

\newpage

\section{The detailed construction}

Let ${\cal G}$ be  a simplicial group with effective homology. This means that 
a  homotopy equivalence:
$$\diagram{
  & \hat{C}_* \cr
 \swarrow\nearrow & & \searrow\nwarrow \cr
{\cal C}_*({\cal G})  & & EG_* \cr
          }$$

is provided, where  the chain complex $EG_*$ is effective 
and must be con\-si\-de\-red as describing
the homology of ${\cal C}_*({\cal G})$. This scheme includes the case where
${\cal C}_*({\cal G})$ itself is  effective; without any other information, 
the program  constructs automatically a trivial homotopy equivalence.
\par
Now, if we apply the  {\em Bar} functor to this homotopy equivalence
we obtain the homotopy equivalence $H_R$:
$$\diagram{
  &{\widetilde{Bar}}^{\hat{ C}_*} \cr
 \swarrow\nearrow & & \searrow\nwarrow \cr
Bar^{{\cal C}_*({\cal G})}({\Z}, {\Z})  & & {\widetilde{Bar}}^{EG_*}({\Z}, {\Z}) \cr
          }$$

in which the $\widetilde{Bar}$'s are Bars constructions with respect to 
the $A_\infty$ structure on $\hat{C}_*$ and $EG_*$ defined by
the initial homotopy equivalence.
\par
By analogy  with the result of Julio Rubio\footnote{{\bf J.J. Rubio-Garcia}. {\em Homologie
effective des espaces de lacets it\'er\'es: un logiciel}, Th\`ese, Institut Fourier, 1991.}
one may show  that it is possible to construct another homotopy equivalence, $H_L$:
$$\diagram{
  & Bar^{{\cal C}_*({\cal G})}({\cal C}_*(\bar{\cal W}{\cal G})\otimes_t {\cal C}_*({\cal G}),{\Z}) \cr
 \swarrow\nearrow & & \searrow\nwarrow \cr
 {\cal C}_*(\bar{\cal W}{\cal G})  & & Bar^{{\cal C}_*({\cal G})}({\Z}, {\Z}) \cr
          }$$
Both reductions of $H_L$ are obtained by the basic perturbation lemma, as explained
in the following section.
The composition of both homotopy e\-qui\-va\-len\-ces, $H_L$ and $H_R$, makes the link between $\bar{\cal W}{\cal G}$,  
the classifying space of  the simplicial group ${\cal G}$
and the effective  right bottom  chain complex of $H_R$.

\subsection {Obtaining the left reduction $H_L$}

The homotopy equivalence  $H_L$ is obtained by a sequence of intermediate constructions
based mainly upon two applications of the basic perturbation lemma.
We are led to consider the two following modules:
\begin{itemize}
\item ${\cal C}_*({\cal G})$ module on itself, with the canonical product.
$${\cal C}_*({\cal G})\otimes {\cal C}_*({\cal G})\stackrel {\Pi}{\longrightarrow} {\cal C}_*({\cal G}).$$
\item ${\check{\cal C}}_*({\cal G})$, module on ${\cal C}_*({\cal G})$, with a ``trivial''
product 
$${\check{\cal C}}_*({\cal G})\otimes{\cal C}_*({\cal G}) \stackrel {\check{\Pi}}{\longrightarrow} 
{\check{\cal C}}_*({\cal G})$$
defined by
$$\sigma \otimes \tau \longmapsto \sigma . \varepsilon(\tau),$$
where $\varepsilon$ is the traditional augmentation of ${\cal C}_*({\cal G})$.
\end{itemize}
Now, we consider the set of the followings Bars
(where $u$ means {\em ``untwisted''} and $t$ {\em ``twisted''}):
\begin{eqnarray*}
{\rm Hat}_{uu} &=& Bar^{{\cal C}_*({\cal G})}({\cal C}_*(\bar{\cal W}{\cal G})\otimes {\check{\cal C}}_*({\cal G}),{\Z}),\\
{\rm Hat}_{ut} &=& Bar^{{\cal C}_*({\cal G})}({\cal C}_*(\bar{\cal W}{\cal G})\otimes {\cal C}_*({\cal G}),{\Z}),\\
{\rm Hat}_{tu} &=& Bar^{{\cal C}_*({\cal G})}({\cal C}_*(\bar{\cal W}{\cal G})\otimes_t{\check{\cal C}}_*({\cal G}),{\Z}),\\
{\rm Hat}_{tt} &=& Bar^{{\cal C}_*({\cal G})}({\cal C}_*(\bar{\cal W}{\cal G})\otimes_t {\cal C}_*({\cal G}),{\Z}).
\end{eqnarray*}

One may always say that ${\rm Hat}_{ut}$ is obtained from ${\rm Hat}_{uu}$ by a perturbation $\delta_r$ ($r$: right)
induced by the discrepancy between the products
in ${\check{\cal C}}_*({\cal G})$ and ${\cal C}_*({\cal G})$ 
and that ${\rm Hat}_{tu}$ is obtained from ${\rm Hat}_{uu}$
by a perturbation $\delta_l$ ($l$: left) induced by the twisted tensor product $\otimes_t$.
After that,
${\rm Hat}_{tt}$ is obtained from ${\rm Hat}_{tu}$ by the perturbation $\delta_r$ as well as, by commutativity,
from ${\rm Hat}_{ut}$ by the perturbation $\delta_l$. \par
This is shown in the following diagram (here, the arrows are not reductions, but denote
the  perturbations between  the differential morphisms):
$$\diagram{ 
  & {\rm Hat}_{uu} \cr
 \delta_l \swarrow  & & \searrow \delta_r \cr
{\rm Hat}_{tu}  & &  {\rm Hat}_{ut} \cr
 \delta_r \searrow  & & \swarrow \delta_l \cr
  & {\rm Hat}_{tt} \cr
          }$$
The underlying graded modules  ${\rm Hat}_{uu}$, ${\rm Hat}_{ut}$, ${\rm Hat}_{tu}$ and ${\rm Hat}_{tt}$ are the same
and the program keeps ${\rm Hat}_{uu}$ as the underlying graded module for all the chain complexes.
The  differential perturbations are given by the formulas:
\begin{eqnarray*}
\delta_l[(w\otimes {\check g}_0)\otimes (g_1 \otimes\cdots\otimes g_n)] &=&
[d_{\otimes t}(w\otimes {\check g}_0) - d_\otimes (w\otimes {\check g}_0)] \otimes (g_1 \otimes\cdots\otimes g_n), \\
\delta_r[(w\otimes {\check g}_0)\otimes (g_1 \otimes\cdots\otimes g_n)] &=& 
w \otimes(g_0 . g_1) \otimes g_2 \otimes\cdots\otimes  g_n,
\end{eqnarray*}
where ${\check g}_0 \in {\check {\cal G}}_0,\, g_i \in {\cal G},\, w \in \bar{\cal W}{\cal G}.$
\par
Now, on the other hand, we know that there exists a reduction  
$${\rm Hat}_{ut} = Bar^{{\cal C}_*({\cal G})}({\cal C}_*(\bar{\cal W}{\cal G})\otimes 
{\cal C}_*({\cal G}),{\Z}) \Longrightarrow {\cal C}_*(\bar{\cal W}{\cal G}),$$ 
so,  perturbing this reduction by $\delta_l$, one obtains the Rubio reduction 
$${\rm Hat}_{tt}=Bar^{{\cal C}_*({\cal G})}({\cal C}_*(\bar{\cal W}{\cal G})\otimes_t 
{\cal C}_*({\cal G}),{\Z}) \Longrightarrow {\cal C}_*(\bar{\cal W}{\cal G}).$$
On the other hand, we know also that there exists a reduction 
$${\rm Hat}_{tu} = Bar^{{\cal C}_*({\cal G})}({\cal C}_*(\bar{\cal W}{\cal G})\otimes_t 
{\check{\cal C}}_*({\cal G}),{\Z}) \Longrightarrow
Bar^{{\cal C}_*({\cal G})}({\Z}, {\Z}),$$ 
so, perturbing this reduction by $\delta_r$, one obtains the reduction 
$${\rm Hat}_{tt}=Bar^{{\cal C}_*({\cal G})}({\cal C}_*(\bar{\cal W}{\cal G})\otimes_t 
{\cal C}_*({\cal G}),{\Z}) \Longrightarrow Bar^{{\cal C}_*({\cal G})}({\Z}, {\Z}).$$
Finally, we have obtained the wished left homotopy equivalence $H_L$:
$$\diagram{
  & Bar^{{\cal C}_*({\cal G})}({\cal C}_*(\bar{\cal W}{\cal G})\otimes_t {\cal C}_*({\cal G}),{\Z}) \cr
    \swarrow\nearrow & & \searrow\nwarrow \cr
 {\cal C}_*(\bar{\cal W}{\cal G})  & & Bar^{{\cal C}_*({\cal G})}({\Z}, {\Z}) \cr
          }$$

\subsection {The useful functions}

For the applications, the only function that the user must know is the function {\tt classifying-space-efhm}
which builds the final homotopy equivalence. But for the
interested user, we give nevertheless a short description of all the functions
involved in the described process.

\vskip 0.30cm
{\parindent=0mm
{\leftskip=5mm
{\tt classifying-space-efhm} {\em smgr}  \hfill {\em [Function]} \par}
{\leftskip=15mm
From the simplicial group ${\cal G}$ ($0$-reduced) with effective homology (here the argument {\em smgr}), build
a homotopy equivalence giving an effective homology version  of the classifying space $\bar{\cal W}{\cal G}$ of ${\cal G}$. 
This homotopy
equivalence will be used by the homology function to compute the homology groups. In fact, due to
the slot-unbound mechanism of CLOS, this function will be automatically called, as soon
as the user requires a homology group for a classifying space. \par}
{\leftskip=5mm
{\tt cs-hat-u-u} {\em smgr}  \hfill {\em [Function]} \par}
{\leftskip=15mm
Return the chain complex 
$Bar^{{\cal C}_*({\cal G})}({\cal C}_*(\bar{\cal W}{\cal G})\otimes {\check{\cal C}}_*({\cal G}),{\Z})$. Because of
the particular structure of ${\check{\cal C}}_*({\cal G})$, this chain complex is nothing but
$[{\cal C}_*(\bar{\cal W}{\cal G})\otimes {\check{\cal C}}_*({\cal G})]\otimes Bar^{{\cal C}_*({\cal G})}({\Z},{\Z}).$
\par}
{\leftskip=5mm
{\tt cs-hat-right-perturbation} {\em smgr}  \hfill {\em [Function]} \par}
{\leftskip=15mm
Return the morphism corresponding to the differential perturbation $\delta_r$, induced by
the discrepancy between the respective products in  ${\check{\cal C}}_*({\cal G})$ and 
in ${\cal C}_*({\cal G})$. \par}
{\leftskip=5mm
{\tt cs-hat-u-t} {\em smgr}  \hfill {\em [Function]} \par}
{\leftskip=15mm
Return the chain complex
$Bar^{{\cal C}_*({\cal G})}({\cal C}_*(\bar{\cal W}{\cal G})\otimes {\cal C}_*({\cal G}),{\Z})$ by applying
the differential perturbation {\tt hat-right-perturbation} upon
the chain complex {\tt hat-u-u {\em smgr}}. This is realized by the method
{\tt add}. \par}
{\leftskip=5mm
{\tt cs-hat-t-u} {\em smgr}  \hfill {\em [Function]} \par}
{\leftskip=15mm
Return the chain complex
$Bar^{{\cal C}_*({\cal G})}({\cal C}_*(\bar{\cal W}{\cal G})\otimes_t {\check{\cal C}}_*({\cal G}),{\Z})$. Because of
the particular structure of ${\check{\cal C}}_*({\cal G})$, this chain complex is nothing but
$[{\cal C}_*(\bar{\cal W}{\cal G})\otimes_t {\check{\cal C}}_*({\cal G})]\otimes Bar^{{\cal C}_*({\cal G})}({\Z},{\Z}),$
where the twisted tensor product is the botton chain complex of the Brown reduction
of the fibration of the simplicial group. \par}
{\leftskip=5mm
{\tt cs-hat-left-perturbation} {\em smgr}  \hfill {\em [Function]} \par}
{\leftskip=15mm
Return the morphism corresponding to the differential perturbation $\delta_l$ induced
by the twisted tensor product. This morphism is nothing but the tensor product of two
morphisms: the perturbation morphism by-product of the Brown reduction of the fibration of
the simplicial group and the identity morphism on $Bar^{{\cal C}_*({\cal G})}({\Z},{\Z})$. \par}
{\leftskip=5mm
{\tt cs-left-hmeq-hat} {\em smgr}  \hfill {\em [Function]} \par}
{\leftskip=15mm
Return the chain complex 
${\rm Hat}_{tt}=Bar^{{\cal C}_*({\cal G})}({\cal C}_*(\bar{\cal W}{\cal G})\otimes_t {\cal C}_*({\cal G}),{\Z})$ by 
perturbing the chain complex ${\rm Hat}_{ut}$ by the perturbation $\delta_l$. \par}
{\leftskip=5mm
{\tt cs-pre-left-hmeq-left-reduction} {\em smgr}  \hfill {\em [Function]} \par}
{\leftskip=15mm
Build the reduction
$${\rm Hat}_{tu}=Bar^{{\cal C}_*({\cal G})}({\cal C}_*(\bar{\cal W}{\cal G})\otimes {\cal C}_*({\cal G}),{\Z}) \Longrightarrow {\cal C}_*(\bar{\cal W}{\cal G}).$$ 
\par}
{\leftskip=5mm
{\tt cs-pre-left-hmeq-right-reduction} {\em smgr}  \hfill {\em [Function]} \par}
{\leftskip=15mm
Build the reduction
$${\rm Hat}_{ut}=Bar^{{\cal C}_*({\cal G})}({\check{\cal C}}_*({\cal G})\otimes_t {\cal C}_*(\bar{\cal W}{\cal G}),{\Z}) \Longrightarrow
Bar^{{\cal C}_*({\cal G})}({\Z}, {\Z}).$$ \par}
{\leftskip=5mm
{\tt cs-left-hmeq-left-reduction} {\em smgr}  \hfill {\em [Function]} \par}
{\leftskip=15mm
Build the Rubio reduction 
$$Bar^{{\cal C}_*({\cal G})}({\cal C}_*(\bar{\cal W}{\cal G})\otimes_t {\cal C}_*({\cal G}),{\Z}) \Longrightarrow{\cal C}_*(\bar{\cal W}{\cal G})$$
by perturbing by the perturbation $\delta_r$ the reduction
$${\rm Hat}_{tu} \Longrightarrow {\cal C}_*(\bar{\cal W}{\cal G})$$ 
obtained by the function {\tt pre-left-hmeq-left-reduction}. \par}
{\leftskip=5mm
{\tt cs-left-hmeq-right-reduction} {\em smgr}  \hfill {\em [Function]} \par}
{\leftskip=15mm
Build the reduction 
$$Bar^{{\cal C}_*({\cal G})}({\cal C}_*(\bar{\cal W}{\cal G})\otimes_t {\cal C}_*({\cal G}),{\Z}) \Longrightarrow 
Bar^{{\cal C}_*({\cal G})}({\Z}, {\Z})$$
by perturbing by the perturbation $\delta_l$ the reduction 
$${\rm Hat}_{ut} \Longrightarrow Bar^{{\cal C}_*({\cal G})}({\Z}, {\Z})$$
obtained by the function {\tt cs-pre-left-hmeq-right-reduction}. \par}
{\leftskip=5mm
{\tt cs-left-hmeq} {\em smgr}  \hfill {\em [Function]} \par}
{\leftskip=15mm
Build the homotopy equivalence
$$\diagram{
  & Bar^{{\cal C}_*({\cal G})}({\cal C}_*(\bar{\cal W}{\cal G})\otimes_t {\cal C}_*({\cal G}),{\Z}) \cr
    \swarrow\nearrow & & \searrow\nwarrow \cr
 {\cal C}_*(\bar{\cal W}{\cal G})    & & Bar^{{\cal C}_*({\cal G})}({\Z}, {\Z}) \cr
          }$$
from the above reductions. The function {\tt classifying-space-efhm} composes this left homotopy
equivalence $H_L$ with the homotopy equivalence $H_R$ which is the Bar of a pre-existing
homotopy equivalence (possibly the trivial one) between the module ${\cal G}$ and an effective
version of it, as described at the beginning of this chapter. 
\par}
}

\subsection {Searching homology for classifying spaces}

The \index{searching homology!classifying spaces} origin list of a classifying space object has the form 
{\tt (CLASSIFYING-SPACE {\em smgr})}.
The {\tt search-efhm} method applied to a classifying space object
consists essentially in a call to the function {\tt classifying-space-efhm}
described just above.
{\footnotesize\begin{verbatim}
(defmethod SEARCH-EFHM (classifying-space (orgn (eql 'classifying-space)))
  (declare (type simplicial-set classifying-space))
  (classifying-space-efhm (second (orgn classifying-space))))
\end{verbatim}}
The following Lisp definition of the function  {\tt classifying-space-efhm} shows that 
the process may be recursif if it happens that {\em smrg} is itself a classifying space:
{\footnotesize\begin{verbatim}
(defun CLASSIFYING-SPACE-EFHM (smgr)
  (declare (type simplicial-group smgr))
  (let ((left-hmeq (cs-left-hmeq smgr))
        (right-hmeq (bar (efhm smgr))))
        (declare (type homotopy-equivalence left-hmeq right-hmeq))
        (cmps left-hmeq right-hmeq)))
\end{verbatim}}


\subsection* {Examples}

{\footnotesize\begin{verbatim}
(setf kz1 (k-z-1))  ==>

[K1 Abelian-Simplicial-Group]

(homology kz1 0 10)  ==>

Homology in dimension 0 :

Component Z

Homology in dimension 1 :

Component Z

Homology in dimension 2 :

Homology in dimension 3 :

Homology in dimension 4 :

Homology in dimension 5 :

Homology in dimension 6 :

Homology in dimension 7 :

Homology in dimension 8 :

Homology in dimension 9 :

---done---

(setf bkz1 (classifying-space kz1))  ==>

[K23 Abelian-Simplicial-Group]

(homology bkz1 0 10)  ==>

Homology in dimension 0 :

Component Z

Homology in dimension 1 :

Homology in dimension 2 :

Component Z

Homology in dimension 3 :

Homology in dimension 4 :

Component Z

Homology in dimension 5 :

Homology in dimension 6 :

Component Z

Homology in dimension 7 :

Homology in dimension 8 :

Component Z

Homology in dimension 9 :

---done---

(setf obkz1 (loop-space bkz1))  ==>

[K154 Simplicial-Group]

(homology obkz1 0 6)  ==>

Homology in dimension 0 :

Component Z

Homology in dimension 1 :

Component Z

Homology in dimension 2 :

Homology in dimension 3 :

Homology in dimension 4 :

Homology in dimension 5 :

---done---

\end{verbatim}}

\subsection* {Lisp files concerned in this chapter}

{\tt cs-space-efhm.lisp}, {\tt searching-homology.lisp}.


