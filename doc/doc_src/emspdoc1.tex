\chapter {Eilenberg-Moore spectral sequence I}

\section{Introduction}

This chapter is devoted to the effective homology version of the spectral
sequence of Eilenberg-Moore, in the particular case of loop spaces.
More precisely, let $X$ be a {\bf $1$-reduced} simplicial set with effective homology,
the software {\tt Kenzo} constructs a $0$-reduced simplicial set $GX$ which is
also with effective homology. In particular, if $X$ is $m$-reduced, this
process may be iterated $m$ times, producing an effective homology
version of $G^kX,\,k\leq m$. So, the software builds an object which contains
a complete solution of the Adams problem about the iteration of the
cobar construction. The user is advised to consult the paper
{\em Stable homotopy and iterated loop spaces, pp. 545} by {\bf Gunnar Carlsson} and
{\bf R.James Milgram} in {\em Handbook of Algebraic Topology, pp 545} edited
by {\em I.M.James, North-Holland, 1995}.

\section{The detailed construction}

Let ${\cal C}_*(X)$ be  a coalgebra with effective homology. This means that 
a   homotopy equivalence:
$$\diagram{
  & \hat{C}_* \cr
 \swarrow\nearrow & & \searrow\nwarrow \cr
{\cal C}_*(X)  & & EC_* \cr
          }$$

is provided, where  the chain complex $EC_*$ is effective 
and must be con\-si\-de\-red as describing
the homology of ${\cal C}_*(X)$. This scheme includes the case where
${\cal C}_*(X)$ is itself effective; without any other information, 
the program  constructs automatically a trivial homotopy equivalence.
\par
Now, if we apply the  {\em Cobar} functor to this homotopy equivalence
we obtain the homotopy equivalence $H_R$:
$$\diagram{
  &{\widetilde{Cobar}}^{{\hat C}_*}({\Z},{\Z}) \cr
 \swarrow\nearrow & & \searrow\nwarrow \cr
Cobar^{{\cal C}_*(X)}({\Z}, {\Z})  & & {\widetilde{Cobar}}^{EC_*}({\Z}, {\Z}) \cr
          }$$

in which the $\widetilde{Cobar}$'s are cobars construction with respect to
the $A_\infty$-co\-al\-ge\-bras structure on ${\hat C}_*$ and $EC_*$ defined by
the initial homotopy equivalence.
\par
Secondly, Julio Rubio\footnote{{\bf J.J. Rubio-Garcia}. {\em Homologie
effective des espaces de lacets it\'er\'es: un logiciel}, Th\`ese, Institut Fourier, 1991.}
has shown that it is possible to construct another homotopy equivalence, $H_L$:
$$\diagram{
  & Cobar^{{\cal C}_*(X)}({\Z}, {\cal C}_*(X)\otimes_t {\cal C_*}(GX)) \cr
 \swarrow\nearrow & & \searrow\nwarrow \cr
 {\cal C_*}(GX)  & & Cobar^{{\cal C}_*(X)}({\Z}, {\Z}) \cr
          }$$
Both reductions of $H_L$ are obtained by the basic perturbation lemma, as explained
in the following section.
The composition of both homotopy e\-qui\-va\-len\-ces, $H_L$ and $H_R$, makes the link between $GX$, 
i.e. the loop space of $X$
and the effective right bottom chain complex of $H_R$.

\subsection {Obtaining the left reduction $H_L$}

The homotopy equivalence  $H_L$ is obtained by a sequence of intermediate constructions
based mainly upon two applications of the basic perturbation lemma.
We are led to consider the two following comodules:
\begin{itemize}
\item ${\cal C}_*(X)$ comodule on itself, with the canonical coproduct.
$${\cal C}_*(X)\stackrel {\Delta}{\longrightarrow} {\cal C}_*(X)\otimes {\cal C}_*(X).$$
\item ${\check{\cal C}}_*(X)$, comodule on ${\cal C}_*(X)$, with a ``trivial''
coproduct 
$${\check{\cal C}}_*(X) \stackrel {\check{\Delta}}{\longrightarrow}  {\cal C}_*(X)\otimes{\check{\cal C}}_*(X),\quad
\sigma \rightarrow 1\otimes\sigma,\,\sigma \in {\check{\cal C}}_*(X).$$
\end{itemize}
Now, we consider the set of the followings cobars
(where $u$ means {\em ``untwisted''} and $t$ {\em ``twisted''}):
\begin{eqnarray*}
{\rm Hat}_{uu} &=& Cobar^{{\cal C}_*(X)}({\Z}, {\check{\cal C}}_*(X)\otimes {\cal C_*}(GX)),\\
{\rm Hat}_{ut} &=& Cobar^{{\cal C}_*(X)}({\Z}, {\check{\cal C}}_*(X)\otimes_t {\cal C_*}(GX)),\\
{\rm Hat}_{tu} &=& Cobar^{{\cal C}_*(X)}({\Z}, {\cal C}_*(X)\otimes {\cal C_*}(GX)),\\
{\rm Hat}_{tt} &=& Cobar^{{\cal C}_*(X)}({\Z}, {\cal C}_*(X)\otimes_t {\cal C_*}(GX)).
\end{eqnarray*}

One may always say that ${\rm Hat}_{ut}$ is obtained from ${\rm Hat}_{uu}$ by a perturbation $\delta_r$ ($r$: right)
induced by the twisted tensor product $\otimes_t$ and that ${\rm Hat}_{tu}$ is obtained from ${\rm Hat}_{uu}$
by a perturbation $\delta_l$ ($l$: left) induced by the discrepancy between the coproducts
in ${\check{\cal C}}_*(X)$ and ${\cal C}_*(X)$. After that,
${\rm Hat}_{tt}$ is obtained from ${\rm Hat}_{tu}$ by the perturbation $\delta_r$ as well as, by commutativity,
from ${\rm Hat}_{ut}$ by the perturbation $\delta_l$. \par
This is shown in the following diagram (here, the arrows are not reductions, but denote
the  perturbations between  the differential morphisms):
$$\diagram{ 
  & {\rm Hat}_{uu} \cr
 \delta_l \swarrow  & & \searrow \delta_r \cr
{\rm Hat}_{tu}  & &  {\rm Hat}_{ut} \cr
 \delta_r \searrow  & & \swarrow \delta_l \cr
  & {\rm Hat}_{tt} \cr
          }$$
The underlying graded modules  ${\rm Hat}_{uu}$, ${\rm Hat}_{ut}$, ${\rm Hat}_{tu}$ and ${\rm Hat}_{tt}$ are the same
and the program keeps ${\rm Hat}_{uu}$ as the underlying graded module for all the chain complexes.
The  differential perturbations are given by the formulas:
\begin{eqnarray*}
\delta_r({\bar{c}}_1 \otimes \cdots \otimes {\bar{c}}_n \otimes c \otimes g) &=& 
 [{\bar{c}}_1 \otimes \cdots \otimes {\bar{c}}_n] \otimes [d_{\otimes t}(c \otimes g)-d_\otimes (c \otimes g)], \\
\delta_l({\bar{c}}_1 \otimes \cdots \otimes {\bar{c}}_n \otimes c \otimes g) &=&
{\bar{c}}_1 \otimes \cdots \otimes {\bar{c}}_n \otimes {\bar{\Delta}}c \otimes g, 
\end{eqnarray*}
where ${\bar{c}}_i, c \in {\cal C}_*(X),\, g \in {\cal C_*}(GX))$ and $\bar{\Delta}= \Delta - \check{\Delta}$.
\par
\vskip 0.35cm
Now, on one hand, we know that there exists a reduction  
$${\rm Hat}_{tu}=Cobar^{{\cal C}_*(X)}({\Z}, {\cal C}_*(X)\otimes {\cal C_*}(GX)) \Longrightarrow {\cal C}_*(GX),$$ 
so,  perturbing this reduction by $\delta_r$, one obtains the Rubio reduction 
$${\rm Hat}_{tt}=Cobar^{{\cal C}_*(X)}({\Z}, {\cal C}_*(X)\otimes_t {\cal C_*}(GX)) \Longrightarrow{\cal C}_*(GX).$$
On the other hand, we know also that there exists a reduction 
$${\rm Hat}_{ut}=Cobar^{{\cal C}_*(X)}({\Z}, {\check{\cal C}}_*(X)\otimes_t {\cal C_*}(GX)) \Longrightarrow
Cobar^{{\cal C}_*(X)}({\Z}, {\Z}),$$ 
so, perturbing this reduction by $\delta_l$, one obtains the reduction 
$${\rm Hat}_{tt}=Cobar^{{\cal C}_*(X)}({\Z}, {\cal C}_*(X)\otimes_t {\cal C_*}(GX)) \Longrightarrow 
Cobar^{{\cal C}_*(X)}({\Z}, {\Z}).$$
Finally, we have obtained the wished left homotopy equivalence $H_L$:
$$\diagram{
  & Cobar^{{\cal C}_*(X)}({\Z}, {\cal C}_*(X)\otimes_t {\cal C_*}(GX)) \cr
    \swarrow\nearrow & & \searrow\nwarrow \cr
 {\cal C_*}(GX)  & & Cobar^{{\cal C}_*(X)}({\Z}, {\Z}) \cr
          }$$

\subsection {The useful functions}

For the applications, the only function that the user must know is the function {\tt loop-space-efhm}
which builds the final homotopy equivalence. But for the
interested user, we give nevertheless a short description of all the functions
involved in the described process.

\vskip 0.30cm
{\parindent=0mm
{\leftskip=5mm
{\tt loop-space-efhm} {\em space}  \hfill {\em [Function]} \par}
{\leftskip=15mm
From the space $X$ ($1$-reduced) with effective homology (here the argument {\em space}), build
a homotopy equivalence giving an effective homology version  of the space $GX$. This homotopy
equivalence will be used by the homology function to compute the homology groups. In fact, due to
the slot-unbound mechanism of CLOS, this function will be automatically called, as soon
as the user requires a homology group of a loop-space.
If $X$ is $n$-reduced and if the wished loop space is $\Omega^k X, \, k \leq n$, then 
the process will be automatically  iterated. \par}
{\leftskip=5mm
{\tt ls-hat-u-u} {\em space}  \hfill {\em [Function]} \par}
{\leftskip=15mm
Return the chain complex 
$Cobar^{{\cal C}_*(X)}({\Z}, {\check{\cal C}}_*(X)\otimes {\cal C_*}(GX))$. Because of
the particular structure of ${\check{\cal C}}_*(X)$, this chain complex is nothing but
$Cobar^{{\cal C}_*(X)}({\Z},{\Z})\otimes {\check{\cal C}}_*(X)\otimes {\cal C_*}(GX).$
\par}
{\leftskip=5mm
{\tt ls-hat-left-perturbation} {\em space}  \hfill {\em [Function]} \par}
{\leftskip=15mm
Return the morphism corresponding to the differential perturbation $\delta_l$, induced by
the discrepancy between the coproducts of  ${\check{\cal C}}_*(X)$ and ${\cal C}_*(X)$. \par}
{\leftskip=5mm
{\tt ls-hat-t-u} {\em space}  \hfill {\em [Function]} \par}
{\leftskip=15mm
Return the chain complex
$Cobar^{{\cal C}_*(X)}({\Z}, {\cal C}_*(X)\otimes {\cal C_*}(GX))$ by applying
the differential perturbation {\tt hat-left-perturbation} upon
the chain complex built by the function {\tt hat-u-u}. This is realized by the method
{\tt add}. \par}
{\leftskip=5mm
{\tt ls-hat-u-t} {\em space}  \hfill {\em [Function]} \par}
{\leftskip=15mm
Return the chain complex
$Cobar^{{\cal C}_*(X)}({\Z},{\check{\cal C}}_*(X)\otimes_t {\cal C_*}(GX))$. Because of
the particular structure of ${\check{\cal C}}_*(X)$, this chain complex is nothing but
$Cobar^{{\cal C}_*(X)}({\Z},{\Z})\otimes[{\check{\cal C}}_*(X)\otimes_t {\cal C_*}(GX)],$
where the twisted tensor product is obtained by a call to {\tt twisted-tnsr-prdc} 
(see loop space fibrations chapter). \par}
{\leftskip=5mm
{\tt ls-hat-right-perturbation} {\em space}  \hfill {\em [Function]} \par}
{\leftskip=15mm
Return the morphism corresponding to the differential perturbation $\delta_r$ induced
by the twisted tensor product. This morphism is nothing but the tensor product of two
morphisms: the identity morphism on $Cobar^{{\cal C}_*(X)}({\Z},{\Z})$ and 
the perturbation morphism induced by the twisted tensor product. This last 
morphism is a by-product of the function {\tt szczarba}. \par}
{\leftskip=5mm
{\tt ls-left-hmeq-hat} {\em space}  \hfill {\em [Function]} \par}
{\leftskip=15mm
Return the chain complex 
${\rm Hat}_{tt}=Cobar^{{\cal C}_*(X)}({\Z}, {\cal C}_*(X)\otimes_t {\cal C_*}(GX))$ by 
perturbing the chain complex ${\rm Hat}_{ut}$ by the perturbation $\delta_l$. \par}
{\leftskip=5mm
{\tt ls-pre-left-hmeq-left-reduction} {\em space}  \hfill {\em [Function]} \par}
{\leftskip=15mm
Build the reduction
$${\rm Hat}_{tu}=Cobar^{{\cal C}_*(X)}({\Z}, {\cal C}_*(X)\otimes {\cal C_*}(GX)) \Longrightarrow {\cal C}_*(GX).$$ 
\par}
{\leftskip=5mm
{\tt ls-pre-left-hmeq-right-reduction} {\em space}  \hfill {\em [Function]} \par}
{\leftskip=15mm
Build the reduction
$${\rm Hat}_{ut}=Cobar^{{\cal C}_*(X)}({\Z},{\check{\cal C}}_*(X)\otimes_t {\cal C_*}(GX)) \Longrightarrow
Cobar^{{\cal C}_*(X)}({\Z}, {\Z}).$$ \par}
{\leftskip=5mm
{\tt ls-left-hmeq-left-reduction} {\em space}  \hfill {\em [Function]} \par}
{\leftskip=15mm
Build the Rubio reduction 
$$Cobar^{{\cal C}_*(X)}({\Z}, {\cal C}_*(X)\otimes_t {\cal C_*}(GX)) \Longrightarrow{\cal C}_*(GX)$$
by perturbing by the perturbation $\delta_r$ the reduction
$${\rm Hat}_{tu} \Longrightarrow {\cal C}_*(GX)$$ 
obtained by the function {\tt pre-left-hmeq-left-reduction}. \par}
{\leftskip=5mm
{\tt ls-left-hmeq-right-reduction} {\em space}  \hfill {\em [Function]} \par}
{\leftskip=15mm
Build the reduction 
$$Cobar^{{\cal C}_*(X)}({\Z}, {\cal C}_*(X)\otimes_t {\cal C_*}(GX)) \Longrightarrow 
Cobar^{{\cal C}_*(X)}({\Z}, {\Z})$$
by perturbing by the perturbation $\delta_l$ the reduction 
$${\rm Hat}_{ut} \Longrightarrow Cobar^{{\cal C}_*(X)}({\Z}, {\Z})$$
obtained by the function {\tt pre-left-hmeq-right-reduction}. \par}
{\leftskip=5mm
{\tt ls-left-hmeq} {\em space}  \hfill {\em [Function]} \par}
{\leftskip=15mm
Build the homotopy equivalence
$$\diagram{
  & Cobar^{{\cal C}_*(X)}({\Z}, {\cal C}_*(X)\otimes_t {\cal C_*}(GX)) \cr
    \swarrow\nearrow & & \searrow\nwarrow \cr
 {\cal C_*}(GX)  & & Cobar^{{\cal C}_*(X)}({\Z}, {\Z}) \cr
          }$$
from the above reductions. The function {\tt loop-space-efhm} composes this left homotopy
equivalence $H_L$ with a homotopy equivalence $H_R$ which is the cobar of a pre-existing
homotopy equivalence (possibly the trivial one) between the coalgebra $X$ and an effective
version of it, as described at the beginning of this chapter. The Lisp definition is given
in the next subsection.
 \par}
}
\newpage

\subsection {Searching homology for loop spaces}

The \index{searching homology!loop spaces} origin list of a loop space object has the form {\tt (LOOP-SPACE {\em space})}.
The {\tt search-efhm} method applied to a loop space object
consists essentially in a call to the function {\tt loop-space-efhm}
described in this chapter.
{\footnotesize\begin{verbatim}
(defmethod SEARCH-EFHM (loop-space (orgn (eql 'loop-space)))
  (declare (type simplicial-set loop-space))
  (loop-space-efhm (second (orgn loop-space))))
\end{verbatim}}
The lisp definition of the function {\tt loop-space-efhm} shows clearly
the possible recursivity of the process is {\em space} is itself
a loop space.
{\footnotesize\begin{verbatim}
(defun LOOP-SPACE-EFHM (space)
  (declare (type simplicial-set space))
  (let ((left-hmeq (ls-left-hmeq space))
        (right-hmeq (cobar (efhm space))))
    (declare (type homotopy-equivalence left-hmeq right-hmeq))
    (cmps left-hmeq right-hmeq)))
\end{verbatim}}

\subsection* {Examples}

Let us show first how to get the homology groups of some known loop spaces, 
$\Omega {\cal S}^2$, $\Omega^3 {\cal S}^4$, $\Omega^2{\rm Moore}({\Z}/{2{\Z}},3)$.
{\footnotesize\begin{verbatim}
(setf s2 (sphere 2))  ==>

[K1 Simplicial-Set]

(setf os2 (loop-space s2))  ==>

[K6 Simplicial-Group]

(dotimes (k 10) (homology os2 k))  ==>

Homology in dimension 0 :

Component Z

Homology in dimension 1 :

Component Z

Homology in dimension 2 :

Component Z

Homology in dimension 3 :

Component Z

Homology in dimension 4 :

Component Z

Homology in dimension 5 :

Component Z

Homology in dimension 6 :

Component Z

Homology in dimension 7 :

Component Z

Homology in dimension 8 :

Component Z

Homology in dimension 9 :

Component Z

---done---

(cat-init)

(setf s4 (sphere 4))  ==>

[K1 Simplicial-Set]

(setf ooos4 (loop-space s4 3))  ==>

[K30 Simplicial-Group]

(dotimes (k 6) (homology ooos4 k))  ==>

Homology in dimension 0 :

Component Z

Homology in dimension 1 :

Component Z

Homology in dimension 2 :

Component Z/2Z

Homology in dimension 3 :

Component Z/2Z

Homology in dimension 4 :

Component Z/3Z

Component Z/2Z

Component Z

Homology in dimension 5 :

Component Z/3Z

Component Z/2Z

Component Z

---done---

(setf moore-23 (moore 2 3))  ==>

[K42 Simplicial-Set]

(setf oo-moore-23 (loop-space moore-23 2))  ==>

[K59 Simplicial-Group]

(dotimes (k 6) (homology oo-moore-23 k))  ==>

Homology in dimension 0 :

Component Z

Homology in dimension 1 :

Component Z/2Z

Homology in dimension 2 :

Component Z/2Z

Homology in dimension 3 :

Component Z/4Z

Component Z/2Z

Homology in dimension 4 :

Component Z/2Z

Component Z/2Z

Component Z/2Z

Homology in dimension 5 :

Component Z/2Z

Component Z/2Z

Component Z/2Z

Component Z/2Z

---done---

\end{verbatim}}

\newpage

\section {The solution of the Adams and Carlsson problem}

Let\index{Adams's problem} $X$ an $n$-reduced simplicial set which is not a suspension. 
The Adams and Carlsson
problem\footnote{{\bf Gunnar Carlsson, R.James Milgram.} {\em Stable homotopy and iterated loop spaces}  in 
{\em Handbook of Algebraic Topology, pp 545} edited by {\em I.M.James, North-Holland, 1995}.}
asks for a finite CW-complex modelizing the iterated loop space $\Omega^n X$. We show in this section
how the {\tt Kenzo} program builds the relevant solution.\par
Let us consider for instance the truncated projective space $X=P^\infty {\R}/ P^3 {\R}$ (a non-suspended space)
and let us build $\Omega^3 X$.
{\footnotesize\begin{verbatim}
(setf p4 (R-proj-space 4))  ==>

[K1 Simplicial-Set]

(setf o3p4 (loop-space p4 3))  ==>

[K30 Simplicial-Group]

(setf ecc (rbcc (efhm o3p4)))  ==>

[K390 Chain-Complex]
\end{verbatim}}
The cellular homology of $\Omega^3 X$ is now known through the symbol {\tt ecc}. Note that $360$ objects
($390-30$) have been built to get the final complex {\tt ecc}. From the printing of the length of the
basis in dimensions $0$ to $5$,
{\footnotesize\begin{verbatim}
(dotimes (i 6) (print (length (basis ecc i))))  ==>

1 
1 
2 
5 
13 
33 
\end{verbatim}}
we see that the required CW-complex can be constructed using $1$ $0$-cell, $1$ $1$-cell, $2$ $2$-cells,
$5$ $3$-cells, $13$ $4$-cells and $33$ $5$-cells. Let us print the incidence matrix between
the $5$ and the $4$-cells:
\newpage
{\footnotesize\begin{verbatim}
(chcm-mat ecc 5)  ==>

========== MATRIX 13 lines + 33 columns =====

L1=[C1=-2]
L2=[C1=-1]
L3=[C1=-4][C2=1][C3=-1][C4=-2]
L4=[C2=1][C3=-1][C6=2]
L5=[C1=6][C4=1][C6=1]
L6=[C1=4][C4=4][C6=4][C7=3]
L7=[C1=4][C12=-2][C14=2]
L8=[C1=6][C4=1][C6=1]
L9=[C1=4][C4=4][C6=4][C7=3]
L10=[C8=4][C10=1][C11=-1][C14=-4][C15=-2][C20=-2]
L11=[C1=4][C8=4][C10=1][C11=-1][C16=-4][C18=-1][C19=1][C23=-2]
L12=[C12=4][C13=2][C16=-4][C18=-1][C19=1][C27=-2]
L13=[C1=-1][C20=4][C21=2][C23=-4][C24=-2][C27=4][C28=2]

========== END-MATRIX
---done---
\end{verbatim}}
For instance, we see that the first $5$-cell,
{\footnotesize\begin{verbatim}
(first (basis ecc 5))  ==>

<<AlLp[5 <<AlLp[6 <<AlLp[7 8]>>]>>]>>
\end{verbatim}}
is glued to the $4$-cells \#5 and \#8 by attaching  maps of degree $6$.
The $5$-cell \#4 is also glued to the $4$-cell \#5 by an attaching map 
of degree $1$, etc.

\subsection* {Lisp files concerned in this chapter}

{\tt lp-space-efhm.lisp}, {\tt searching-homology.lisp}.


