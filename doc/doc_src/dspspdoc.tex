\chapter {Disk pasting and suspension}

\section{Introduction}

Let\index{attaching maps} us recall the technique of space construction by 
{\em attaching maps}\footnote {{\bf Marvin Greenberg} in
{\em Lectures on Algebraic Topology.} Benjamin Inc, 1967.}. Let $A$ be a subspace of a space $X$ and a map
$f$ of $A$ into a space $Y$. In the space $X \coprod Y$, let us identify each $x$ in $A$ with its image
$f(x)$ in $Y$. The quotient space $Z=X\bigcup_f Y$ of the space $X\coprod Y$ by the equivalence
relation that the identification above determines, is called the {\em adjunction space} of the system
$(X \supset A), (Y\supset f(A))$. The quotient map $g: X\coprod Y \rightarrow Z$ sends $Y$ 
onto a subspace of $Z$. 
\par
In this section, we shall take as pair $(X,A)$ the pair $(D^n, S^{n-1})$, where $D^n$ is the 
unit disc of ${\R}^n$ and  $S^{n-1}$ is the sphere of dimension ${n-1}$, boundary of $D^n$. In this case,
the space $Z=D^n\bigcup_f Y$ is said to be obtained from $Y$ by {\em attaching an n--disc via} $f$, or
{\em disk pasting}.\index{disk pasting}

\section{The functions for  disk pasting}

The {\bf Kenzo} program implements the previous technique of  space construction, via simplicial sets.
We know that $D^n$ is homeomorphic to the standard simplex $\Delta^n$. The method used
by the implementor is the following:
\par
Starting from a simplicial set {\em ss} representing $Y$, build a new one having the same simplices
as {\em ss} and in dimension $n$, a new added simplex described by the user. This description
consists of: \par
-- a name (a symbol) for this new generator,\par
-- the list of all its $n+1$ faces. 
\newpage
{\parindent=0mm
{\leftskip=5mm 
{\tt disk-pasting} {\em smst dmns new faces }\hfill{\em [Function]} \par}
{\leftskip=15mm 
Construct, from the  simplicial set  {\em smst}, a new simplicial set
by attaching the unit disc of dimension {\em dmns}. The argument {\em faces} is the
list of  $dmn+1$  simplices corresponding to the faces of the new simplex 
attached to {\em smst}. These simplices are, in principle, written as abstract simplices 
({\em absm}), nevertheless it is possible to write them as geometric simplices ({\em gmsm}). In this case, 
the corresponding geometric simplices must have the correct dimension. The program will transform the list of
{\em gmsm}'s into a list of {\em absm}'s.  The argument {\em new} allows the user to name this new added simplex. 
The program verifies the coherence of the addition of the new simplex by calling internally the function
{\tt check-faces}.
\par}
{\leftskip=5mm 
{\tt chcm-disk-pasting} {\em chcm dmns new  bndr}\hfill{\em [Function]} \par}
{\leftskip=15mm 
Construct, from the chain complex  {\em chcm} (chain groups ${\cal C}_*$), a new chain complex 
(chain groups ${\cal C}_*'$), where:
$$C'_p=C_p,\quad \forall p \not= dmn,$$
$$C'_{dmn}= C_{dmn} \oplus {\Z}\, new.$$
The symbol {\tt new} is the name given by the user to a new generator in dimension $dmns$ and
{\em bndr} is a combination of degree $dmns-1$ representing the  boundary of the
generator {\em new}. Of course, one must have $d_{dmns-1}(bndr)=0.$ \par}
{\leftskip=5mm 
{\tt hmeq-disk-pasting} {\em hmeq dmns new bndr {\tt \&key} new-lbcc } \hfill{\em [Function]}\par}
{\leftskip=15mm 
Construct a homotopy equivalence by attaching  the  generator {\em new} of dimension {\em dmns} 
with boundary {\em bndr} to the left bottom chain complex of the homotopy equivalence {\em hmeq}. 
This modification of the left bottom chain complex is propagated along the whole new homotopy equivalence.
The key parameter {\em new-lbcc} is for internal use.
\par}
}

\newpage

\subsection* {Example}

Let us begin by a trivial example corresponding to the well known following result:
if $f$ maps $S^{n-1}$ onto a point $Y$, then by disk pasting we obtain a space homeomorphic to $S^n$. 
Here, the point $Y$ will be represented by the trivial simplicial set:

{\footnotesize\begin{verbatim}
 (setf s0 (build-finite-ss '(*)))  ==>

Checking the 0-simplices...
[K1 Simplicial-Set]
\end{verbatim}}
In the identification process, the 1-simplices boundaries of the 2-simplex $\Delta^2$
are applied on the 0--degeneracy of the base point $*$.
{\footnotesize\begin{verbatim}
(setf s2 (disk-pasting s0 
                        2 
                       's2 
                       (list (absm 1 '*) (absm 1 '*) (absm 1 '*))  )) 

[K6 Simplicial-Set]

(show-structure s2 2)  ==>

Dimension = 0 :

        Vertices :  (*)

Dimension = 1 :

Dimension = 2 :

        Simplex : S2

                Faces : (<AbSm 0 *> <AbSm 0 *> <AbSm 0 *>)
\end{verbatim}}
More interesting is the construction of the successive projectives spaces
$P^i{\R}$. Let us begin by $P^1{\R}$. We start from a point (the simplicial set {\tt s0}) 
and we identify the two boundary
points of $\Delta^1$ to the base point of {\tt s0}. The new simplex to be added to {\tt s0}
is of dimension $1$ and its $2$ faces  are the base point itself. It is well known that
the resulting corresponding  space is homeomorphic to the circle $S^1$.
{\footnotesize\begin{verbatim}
(setf p1r (disk-pasting s0 1 'd1 (list (absm 0 '*)(absm 0 '*))))  ==>

[K34 Simplicial-Set]

(show-structure p1r 1)  ==>

Dimension = 0 :

        Vertices :  (*)

Dimension = 1 :

        Simplex : D1

                Faces : (<AbSm - *> <AbSm - *>)

(dotimes (i 3) (homology p1r i))  ==>

Homology in dimension 0 :

Component Z

Homology in dimension 1 :

Component Z

Homology in dimension 2 :

---done---
\end{verbatim} }

The projective spaces $P^n{\R}$ of higher degree may be constructed by the following iterative
rule. Suppose we have built a simplicial set corresponding to $P^{n-1}{\R}$ and that we have
named the  additional simplex $d_{n-1}$. Suppose also that $d_{n-2}$ has been previously defined.
($d_0$ is the base point). 
Then the simplicial set corresponding to $P^n{\R}$ is obtained by pasting $\Delta^n$ with the following
description of the additional simplex: 
\begin{itemize}
\item the name of the new simplex is  $d_n$,
\item the  faces of $d_n$ are:
\begin{eqnarray*}
\partial_0 d_n & = & d_{n-1},\\
\partial_1 d_n & = & \eta_0 d_{n-2},\\
\partial_2 d_n & = & \eta_1 d_{n-2},\\
\cdots         & = & \cdots , \\
\partial_{n-1} d_n & = & \eta_{n-2} d_{n-2},\\
\partial_n d_n & = & d_{n-1}.
\end{eqnarray*}
\end{itemize}
So, applying this rule, we may construct a  model for  $P^i{\R},\, i=2,3,4$ and verify  the well known results
about their respective homology groups.
{\footnotesize\begin{verbatim}
(setf p2r (disk-pasting p1r 2 'd2
                        (list (absm 0 'd1)
                              (absm 1 '*)
                              (absm 0 'd1)) )) ==>

[K42 Simplicial-Set]

(show-structure p2r 2)  ==>

Dimension = 0 :

        Vertices :  (*)

Dimension = 1 :

        Simplex : D1

                Faces : (<AbSm - *> <AbSm - *>)

Dimension = 2 :

        Simplex : D2

                Faces : (<AbSm - D1> <AbSm 0 *> <AbSm - D1>)

(dotimes (i 3) (homology p2r i))  ==>

Homology in dimension 0 :

Component Z

Homology in dimension 1 :

Component Z/2Z

Homology in dimension 2 :

---done---

(setf p3r (disk-pasting p2r 3 'd3
                        (list (absm 0 'd2) 
                              (absm 1 'd1)
                              (absm 2 'd1)
                              (absm 0 'd2))))  ==>

[K56 Simplicial-Set]

(show-structure p3r 3)  ==>

Dimension = 0 :

        Vertices :  (*)

Dimension = 1 :

        Simplex : D1

                Faces : (<AbSm - *> <AbSm - *>)

Dimension = 2 :

        Simplex : D2

                Faces : (<AbSm - D1> <AbSm 0 *> <AbSm - D1>)

Dimension = 3 :

        Simplex : D3

                Faces : (<AbSm - D2> <AbSm 0 D1> <AbSm 1 D1> <AbSm - D2>)

(dotimes (i 4)(homology p3r i))  ==>

Homology in dimension 0 :

Component Z

Homology in dimension 1 :

Component Z/2Z

Homology in dimension 2 :

Homology in dimension 3 :

Component Z


(setf p4r (disk-pasting p3r 4 'd4 
                        (list (absm 0 'd3)
                              (absm 1 'd2)
                              (absm 2 'd2) 
                              (absm 4 'd2)
                              (absm 0 'd3))))  ==>

[K70 Simplicial-Set]

(dotimes (i 5) (homology p4r i))  ==>

Homology in dimension 0 :

Component Z

Homology in dimension 1 :

Component Z/2Z

Homology in dimension 2 :

Homology in dimension 3 :

Component Z/2Z

Homology in dimension 4 :

---done---
\end{verbatim}}

Let us give now some examples with the function {\tt chcm-disk-pasting}. We are going to work with 
the unit chain complex (obtained by the function {\tt z-chcm}, see chapter 1), 
having a unique non null component, 
namely a $\Z$--module of degree $0$ generated by the unique generator {\tt :Z-gnrt}. 
{\footnotesize\begin{verbatim}
(setf *ccz* (z-chcm))  ==>

[K84 Chain-Complex]
\end{verbatim}}
Using the function {\tt chcm-disk-pasting}, let us construct from {\tt *ccz*}
a new chain complex  having in dimension $1$ a unique  new generator {\tt gen-1} 
whose boundary  is 5 times the unique generator in dimension $0$. In this new chain complex the
homology group in dimension $0$ is  ${\Z}/5\,{\Z}$.
{\footnotesize\begin{verbatim}
(setf newcc-0 (chcm-disk-pasting *ccz* 1 'gen-1 (cmbn 0 5 :Z-gnrt)))  ==>

[K86 Chain-Complex]

(chcm-homology-gen newcc-0 0)  ==>

Homology in dimension 0 :

Component Z/5Z

Generator :

1     *  :Z-GNRT

(chcm-homology-gen newcc-0 1)  ==>

Homology in dimension 1 :

---done---
\end{verbatim}}
Let us do the same with a generator whose boundary is  $165$ times the ge\-ne\-ra\-tor {\tt :Z-gnrt}:
{\footnotesize\begin{verbatim}
(setf newcc-1 (chcm-disk-pasting *ccz* 1 'gen-1 (cmbn 0 165 :Z-gnrt))) ==>

[K88 Chain-Complex]

(chcm-homology-gen newcc-1 0)  ==>

Homology in dimension 0 :

Component Z/165Z

Generator :

1     *  :Z-GNRT
\end{verbatim}}
Now from the chain complex {\tt newcc-1}, let us build a new one, {\tt newcc-2},
by adding in dimension $1$ a second generator whose boundary  is $182$ times
the generator {\tt :Z-gnrt}. The homology group in dimension $0$ of this new chain complex
is  the null group, since $165$ and $182$ are relatively prime integers.
{\footnotesize\begin{verbatim}
(setf newcc-2 (chcm-disk-pasting newcc-1 1 'gen-2 (cmbn 0 182 :Z-gnrt)))  ==>

[K92 Chain-Complex]

(chcm-homology-gen newcc-2 0)  ==>

Homology in dimension 0 :

---done---

(chcm-homology-gen newcc-2 1)  ==>

Homology in dimension 1 :

Component Z

Generator :

-165  *  GEN-2
182   *  GEN-1
\end{verbatim}}
As a second example, let us start from $\Omega^1(S^3)$.
{\footnotesize\begin{verbatim}
(setf s3 (sphere 3))  ==>

[K94 Simplicial-Set]

(setf os3 (loop-space s3))  ==>

[K99 Simplicial-Group]
\end{verbatim}}
In the following instruction, we locate in the symbol {\tt fund-simp} the canonical generator of 
$\pi_2 (\Omega^1S^3)$, that is the $2$--simplex coming from the original sphere:
{\footnotesize\begin{verbatim}
(setf fund-simp (absm 0 (loop3 0 's3 1)))  ==>

<AbSm - <<Loop[S3]>>>
\end{verbatim}}
We need also the $2$--degeneracy of the base point of the loop space:
{\footnotesize\begin{verbatim}
(setf null-simp (absm 3 +null-loop+))  ==>

<AbSm 1-0 <<Loop>>>
\end{verbatim}}
We may  build now a new  object by pasting a disk $D^3$ as indicated by the 
following call:
{\footnotesize\begin{verbatim}
(setf dos3 (disk-pasting os3 3 '<D3> 
                        (list fund-simp null-simp fund-simp null-simp)))  ==>

[K212 Simplicial-Set]

(homology dos3 2)  ==>

Homology in dimension 2 :

Component Z/2Z

(homology dos3 3)  ==>

Homology in dimension 3 :

---done---
\end{verbatim}}
Then, let us  build the  loop space of the object {\tt dos3} and show the ho\-mo\-lo\-gy
in dimension $5$:
{\footnotesize\begin{verbatim}
(setf odos3 (loop-space dos3))  ==>

[K230 Simplicial-Group]

(homology odos3 5)  ==>

Homology in dimension 5 :

Component Z/2Z

Component Z/2Z

Component Z/2Z

Component Z/2Z

Component Z/2Z

Component Z/2Z

Component Z
\end{verbatim}}

\subsection {Searching Homology for objects created by disk pasting}

The comment list of an object created by disk pasting\index{searching homology!disk pasting}
contains various informations as shown by the simple following example
{\footnotesize\begin{verbatim}
(setf s0 (build-finite-ss '(*)))  ==>

Checking the 0-simplices...

[K1 Simplicial-Set]

(setf s2 (disk-pasting s0 
                        2 
                       's2 
                       (list (absm 1 '*) (absm 1 '*) (absm 1 '*))  ))  ==>

[K6 Simplicial-Set]

(orgn s2)  ==>

(DISK-PASTING [K1 Simplicial-Set] 2 S2 (<AbSm 0 *> <AbSm 0 *> <AbSm 0 *>))
\end{verbatim}}
In fact, the comment list contains the very arguments used to realize the attaching.
The method retrieves those arguments, gets the homotopy equivalence value of the
{\tt efhm} slot of the old object --creating it if necessary -- then build
a new homotopy equivalence, by calling the function {\tt hmeq-disk-pasting {\em hmeq}} which 
takes in account the attaching.
{\footnotesize\begin{verbatim}
(DEFMETHOD SEARCH-EFHM (smst (orgn (eql 'disk-pasting)))
  (declare (type simplicial-set smst))
  (the homotopy-equivalence
   (destructuring-bind (old-smst dmns new faces) (rest (orgn smst))
       (declare
          (type simplicial-set old-smst)
          (fixnum dmns)
          (symbol new)
          (ignore faces))
       (hmeq-disk-pasting 
               (efhm old-smst)
               dmns new (? smst dmns new)
               :new-lbcc smst))))
\end{verbatim}}
\newpage

\section {The functions for the suspension}

In {\bf Kenzo} the suspension\index{suspension} process is realized by the function {\tt suspension}.
This function has one argument, a reduced  object (chain complex, simplicial set, etc.).
{\bf If the object is not reduced, the result is undefined}.
The CLOS mechanism is used to apply an adequate method for building the suspension of the object.
For a chain complex, a new base point is created with name {\tt :s-bsgn} and the degree of the generators
is increased by $1$. If the suspension process is applied to a homotopy equivalence, then firstly, 
the left bottom chain complex is suspended and secondly, the modification is propagated along 
the  whole homotopy equivalence.
\par
\vskip 0.40cm
{\parindent=0mm
{\leftskip=5mm 
{\tt suspension-cmpr} {\em cmpr}\hfill{\em[Function]} \par}
{\leftskip=15mm 
From the comparison function {\em cmpr}, build a comparison function to compare
two generators of a suspension. \par}
{\leftskip=5mm 
{\tt suspension-basis} {\em basis}\hfill{\em[Function]} \par}
{\leftskip=15mm 
From the function {\em basis} of  a chain complex, build a basis function for
the suspension of this chain complex. In degree $0$, the only generator is {\tt :s-bsgn},
in degree $1$, the basis is void and in degree $k,\,k\ge 2$, the elements of the basis of
the suspension are the elements of the initial chain complex in degree $k-1$. \par}
{\leftskip=5mm 
{\tt suspension-intr-dffr} {\em dffr}\hfill{\em[Function]} \par}
{\leftskip=15mm 
From the lisp differential function {\em dffr} of a chain complex, build the
lisp differential function for the suspension of this chain complex. \par}
{\leftskip=5mm 
{\tt suspension-intr-cprd} {\em cmbn}\hfill{\em[Function]} \par}
{\leftskip=15mm 
Return the result of the  application of the coproduct of a suspension 
(considered as a coalgebra) upon the combination {\em cmbn}. Note that
this function does not need any other argument than the combination. \par}
{\leftskip=5mm 
{\tt suspension-face} {\em face}\hfill{\em[Function]} \par}
{\leftskip=15mm 
From the function face of a simplicial set, build the face function
for the suspension of this simplicial set. \par}
{\leftskip=5mm 
{\tt suspension-intr} {\em mrph}\hfill{\em[Function]} \par}
{\leftskip=15mm 
From a {\bf Kenzo} morphism {\em mrph}, build an internal function 
cor\-res\-pon\-ding to the suspension of the initial morphism. This function
will be applied to a combination assumed to belong to a suspension
of a chain complex. \par}
}
\newpage
{\parindent=0mm
{\leftskip=5mm 
{\tt suspension} {\em obj {\tt \&optional} (n 1)}\hfill{\em[Function]} \par}
{\leftskip=15mm 
Construct a new object, the suspension of the {\bf reduced} object {\em obj}. 
The (geometrical) base point of the suspension of a chain complex
(or simplicial set) is named  {\tt :s-bsgn}. The optional parameter {\em n} (default: $1$)
allows to create in one instruction iterated suspensions. This function calls the adequate
method ({\tt suspension-1}) for the object {\em obj}.\par}
{\leftskip=5mm 
{\tt suspension-1} {\em chcm}\hfill{\em[Method]} \par}
{\leftskip=15mm 
Build the chain complex suspension of the chain complex {\em chcm}, using the basic functions
above, as shown in the following call to {\tt build-chcm}:
{\footnotesize\begin{verbatim}
           (build-chcm
             :cmpr (suspension-cmpr cmpr)
             :basis (suspension-basis basis)
             :bsgn :s-bsgn
             :intr-dffr (suspension-intr-dffr dffr)
             :strt :cmbn
             :orgn `(suspension ,chcm))
\end{verbatim}} 
\par}
{\leftskip=5mm 
{\tt suspension-1} {\em clbg}\hfill{\em[Method]} \par}
{\leftskip=15mm 
Return the coalgebra associated to the suspension of the coalgebra {\em clbg}.
This calls internally the method for the suspension of the chain complex {\em clbg} then  
assigns the function {\tt suspension-intr-cprd} to the value of the slot {\tt cprd}. \par}
{\leftskip=5mm 
{\tt suspension-1} {\em smst}\hfill{\em[Method]} \par}
{\leftskip=15mm 
Return the simplicial set, suspension of the simplicial set {\em smst}.
This function uses the previous basic function {\tt suspension-face}.
The resulting object is also a coalgebra with a coproduct defined by the
function {\tt suspension-intr-cprd}. \par}
{\leftskip=5mm 
{\tt suspension-1} {\em mrph}\hfill{\em[Method]} \par}
{\leftskip=15mm 
Build the suspension of the morphism {\em mrph}. This is a morphism of the same
degree as {\em mrph}. If {\em mrph} is the zero morphism, return the zero morphism
between the suspensions of the source and target of {\em mrph}. If {\em mrph} is the
identity morphim, return the identity morphism upon the suspension of the source
of {\em mrph}. Otherwise, return the morphism built by the following call to
{\tt build-mrph}:
{\footnotesize\begin{verbatim}
              (build-mrph
                 :sorc (suspension (sorc mrph))
                 :trgt (suspension (trgt mrph))
                 :degr (degr mrph)
                 :intr (suspension-intr mrph)
                 :strt :cmbn
                 :orgn `(suspension ,mrph))
\end{verbatim}}
\par}
{\leftskip=5mm 
{\tt suspension-1} {\em rdct}\hfill{\em[Method]} \par}
{\leftskip=15mm 
Build the suspension of the reduction {\em rdct}. If {\em rdct} is the trivial 
reduction upon a chain complex ${\cal C}$, return the trivial reduction upon
the suspension of ${\cal C}$. Otherwise, return the reduction built by the following call to
the function {\tt build-rdct}:
{\footnotesize\begin{verbatim}
              (build-rdct
                 :f (suspension (f rdct))
                 :g (suspension (g rdct))
                 :h (suspension (h rdct))
                 :orgn `(suspension ,rdct))
\end{verbatim}}
\par}
{\leftskip=5mm 
{\tt suspension-1} {\em hmeq}\hfill{\em[Method]} \par}
{\leftskip=15mm 
Build the suspension of the  homotopy equivalence {\em hmeq}. If {\em hmeq} is the trivial 
homotopy equivalence  upon a chain complex ${\cal C}$, return the trivial homotopy equivalence upon
the suspension of ${\cal C}$. O\-ther\-wi\-se, return the homotopy equivalence built by the following call to
the function {\tt build-hmeq}:
{\footnotesize\begin{verbatim}
              (build-hmeq
                 :lrdct (suspension (lrdct hmeq))
                 :rrdct (suspension (rrdct hmeq))
                 :orgn `(suspension ,hmeq))
\end{verbatim}}
\par}
}

\subsection* {Examples}

Firstly we test  the coproduct of the suspension of $\bar\Delta$. In the third
statement, the generators $7$ and $11$ are the binary coded representation 
of the simplices {\tt (0 1 2)} and  {\tt (0 1 3)} in dimension $3$ of 
${\cal S}\bar\Delta$,
{\footnotesize\begin{verbatim}
(suspension-intr-cprd (cmbn 0))  ==>

----------------------------------------------------------------------{CMBN 0}
------------------------------------------------------------------------------

(suspension-intr-cprd (cmbn 0 5 :s-bsgn))  ==>

----------------------------------------------------------------------{CMBN 0}
<5 * <TnPr S-BSGN S-BSGN>>
------------------------------------------------------------------------------

(suspension-intr-cprd (cmbn 3 4 7 5 11))  ==>

----------------------------------------------------------------------{CMBN 3}
<4 * <TnPr S-BSGN 7>>
<5 * <TnPr S-BSGN 11>>
<4 * <TnPr 7 S-BSGN>>
<5 * <TnPr 11 S-BSGN>>
------------------------------------------------------------------------------
\end{verbatim}}

Let us verify now the classical result ${\cal S}(S^n)=S^{n+1}$.
{\footnotesize\begin{verbatim}
(setf s1 (sphere 1))  ==>

[K1 Simplicial-Set]

(setf ss1 (suspension s1))  ==>

[K6 Simplicial-Set]

(orgn ss1)  ==>

(SUSPENSION [K1 Simplicial-Set])

(show-structure ss1 2)  ==>

Dimension = 0 :

        Vertices :  (S-BSGN)

Dimension = 1 :

Dimension = 2 :

        Simplex : S1

                Faces : (<AbSm 0 S-BSGN> <AbSm 0 S-BSGN> <AbSm 0 S-BSGN>)
\end{verbatim}}
For comparison, let us recall the simplicial set model of $S^2$:
{\footnotesize\begin{verbatim}
(show-structure (sphere 2) 2)  ==>

Dimension = 0 :

        Vertices : (*)

Dimension = 1 :

Dimension = 2 :

        Simplex : S2

                Faces : (<AbSm 0 *> <AbSm 0 *> <AbSm 0 *>)
\end{verbatim}}
Let us iterate twice the operation of suspension on $S^2$:
{\footnotesize\begin{verbatim}
(setf ss22 (suspension (sphere 2) 2))  ==>

[K21 Simplicial-Set]

(show-structure ss22 4)  ==>

Dimension = 0 :

        Vertices :  (S-BSGN)

Dimension = 1 :

Dimension = 2 :

Dimension = 3 :

Dimension = 4 :

        Simplex : S2

                Faces : (<AbSm 2-1-0 S-BSGN> <AbSm 2-1-0 S-BSGN> 
                         <AbSm 2-1-0 S-BSGN> <AbSm 2-1-0 S-BSGN> 
                         <AbSm 2-1-0 S-BSGN>)
\end{verbatim}}
To test the suspension of a less simple chain complex, let us take a Moore space that we have seen in the 
simplicial set chapter.
We know that the corresponding  chain complex  is connected. Let us compare the
basis elements, in various degrees, of this chain complex and its suspension:
{\footnotesize\begin{verbatim}
(setf m23 (moore 2 3))  ==>

[K26 Simplicial-Set]

(dotimes (i 7) (print (basis m23 i)))

(*) 
NIL 
NIL 
(M3) 
(N4) 
NIL 
NIL 
NIL

(setf sm23 (suspension m23))  ==>

[K31 Simplicial-Set]

(dotimes (i 7) (print (basis sm23 i)))

(:S-BSGN) 
NIL 
NIL 
NIL 
(M3) 
(N4) 
NIL 
NIL
\end{verbatim}}

Let us   apply  now the function {\tt suspension}  to 
objects built from $S^2$, namely $\Omega^1(S^2)$ and $\Omega^2(S^2)$ and let us have a look
upon the homology groups . For $S^2$, we know that the only
non--null homology groups are in dimension $0$ and $2$:
{\footnotesize\begin{verbatim}
(homology (sphere 2) 0 5)  ==>

Homology in dimension 0 :

Component Z

Homology in dimension 1 :

Homology in dimension 2 :

Component Z

Homology in dimension 3 :

Homology in dimension 4 :

---done---
\end{verbatim}}
For ${\cal S}(S^2)$, the non--null homology groups are of course in dimension $0$ and $3$:
{\footnotesize\begin{verbatim}
(setf ss2 (suspension (sphere 2)))  ==>

[K16 Simplicial-Set]

(homology ss2 0 5)  ==>

Homology in dimension 0 :

Component Z

Homology in dimension 1 :

Homology in dimension 2 :

Homology in dimension 3 :

Component Z

Homology in dimension 4 :

---done---
\end{verbatim}}
Let us verify now that, for the space $\Omega(S^2)$, the homology groups are organized as 
the tensor algebra over one generator in degree $1$:
{\footnotesize\begin{verbatim}
(setf s2 (sphere 2))  ==>

[K11 Simplicial-Set]

(setf os2 (loop-space s2))  ==>

[K44 Simplicial-Group]

(homology os2 0 5)  ==>

Homology in dimension 0 :

Component Z

Homology in dimension 1 :

Component Z

Homology in dimension 2 :

Component Z

Homology in dimension 3 :

Component Z

Homology in dimension 4 :

Component Z

\end{verbatim}}
And, in the suspension of the previous space, we verify that the homology groups
are suspended (in particular, there is no homology in dimension $1$)
{\footnotesize\begin{verbatim}
(setf sos2 (suspension os2))  ==>

[K153 Simplicial-Set]

(homology sos2 0 5)  ==>

Homology in dimension 0 :

Component Z

Homology in dimension 1 :

Homology in dimension 2 :

Component Z

Homology in dimension 3 :

Component Z

Homology in dimension 4 :

Component Z
\end{verbatim}}
At last, taking the loop space of the previous suspension, we recognize the tensor
algebra over the previous tensor algebra, ${\cal H}_*\Omega S^2$. In each dimension, the
number of generators is $2^n$:
{\footnotesize\begin{verbatim}
(setf osos2 (loop-space sos2))  ==>

[K171 Simplicial-Group]

(homology osos2 0 5)  ==>

Homology in dimension 0 :

Component Z

Homology in dimension 1 :

Component Z

Homology in dimension 2 :

Component Z

Component Z

Homology in dimension 3 :

Component Z

Component Z

Component Z

Component Z

Homology in dimension 4 :

Component Z

Component Z

Component Z

Component Z

Component Z

Component Z

Component Z

Component Z
\end{verbatim}}

\subsection {Searching homology for suspensions}

The comment list of a suspension\index{searching homology!suspension} has the 
form {\tt (SUSPENSION {\em suspended-object})}.
The {\tt search-efhm} method applied to a suspension, looks for the value of the 
{\tt efhm} slot of the {\em suspended-object}, (i.e. a homotopy equivalence), then
builds the suspension of this homotopy equivalence (see the method {\tt suspension} applied
to a homotopy equivalence). Of course, this may imply a recursive process.
{\footnotesize\begin{verbatim}
(DEFMETHOD SEARCH-EFHM (suspension (orgn (eql 'suspension)))
  (declare (type chain-complex suspension))
  (suspension (efhm (second (orgn suspension)))))
\end{verbatim}}

\subsection* {Lisp files concerned in this chapter}

{\tt disk-pasting.lisp}, {\tt suspensions.lisp}, {\tt searching-homology.lisp}


