\chapter {Simplicial groups fibrations}

\section {Introduction}

This chapter\index{fibrations!simplicial groups} is analogous to the  Loop spaces fibrations chapter  
where  algorithms have been specially designed for loop spaces. In the present case, we have at our
disposal all the tools developped for the fibrations. Let ${\cal G}$
be a $0$--reduced simplicial group and $\bar{\cal W}{\cal G}$, its
classifying space.  We recall that if ${\cal G}$ is
an Abelian simplicial group then $\bar{\cal W}{\cal G}$ is also an Abelian
simplicial group, otherwise if ${\cal G}$ is non--Abelian, then $\bar{\cal W}{\cal G}$
is only a simplicial set. The software {\tt Kenzo} implements the
{\bf canonical} twisted cartesian product of $\bar{\cal W}{\cal G}$ by  ${\cal G}$, 
denoted ${\cal WG}=\bar{\cal W}{\cal G} \times_\tau  {\cal G}$.
The {\em canonical fibration}
$$ {\cal G}\quad \hookrightarrow\quad  {\cal WG} \hskip 1.2em \longrightarrow \hskip -1.2em \longrightarrow  
   \quad\bar{\cal W}{\cal G},$$
is defined by the following twisting operator:
$$\tau[(g_{n-1}, g_{n-2}, \ldots, g_0)]= g_{n-1}.$$
The total space ${\cal WG}=\bar{\cal W}{\cal G} \times_\tau  {\cal G}$ is acyclic.

\section {The function for the canonical fibration}

{\parindent=0mm
{\leftskip=5mm
{\tt smgr-fibration} {\em smgr} \hfill {\em [Function]} \par}
{\leftskip=15mm
Build the simplicial morphism of degree $-1$ corresponding to the canonical fibration.
The source is the classifying space of the simplicial group {\em smgr} 
(built internally by the function), the target
is {\em smgr} and the internal lisp function value of the slot {\tt :sintr} implements
the twisting operator. The returned object is a fibration. \par}
}

\subsection* {Examples}

Let us take the Abelian simplicial group $K({\Z},1)$ and  build the fibration morphism.
{\footnotesize\begin{verbatim}
(setf k1 (k-z-1))  ==>

[K1 Abelian-Simplicial-Group]

(setf tw (smgr-fibration k1))  ==>

[K25 Fibration]
\end{verbatim}}
If needed, the classifying space may be found in the slot {\tt :sorc} of the fibration:
{\footnotesize\begin{verbatim}
(setf k2 (sorc tw))  ==>

[K13 Abelian-Simplicial-Group]
\end{verbatim}}
We may build now the total space of the fibration. We see that {\tt Kenzo} returns
a Kan simplicial set because the base and fiber spaces are also of
Kan type. Then we test the face and differential operator upon a simplex of degree $4$ of
the total space {\tt tt}.
{\footnotesize\begin{verbatim}
(setf tt (fibration-total tw))  ==>

[K31 Kan-Simplicial-Set]

(setf gmsm (crpr 0 (gbar 4 0 '(10 11 12) 0 '(20 21) 0 '(30) 0 '())
                 0 '(2 4 6 8)))  ==>

<CrPr - <<GBar<- (10 11 12)><- (20 21)><- (30)><- NIL>>> - (2 4 6 8)>

(dotimes (i 5) (print (face tt i 4 gmsm)))  ==>

<AbSm - <CrPr - <<GBar<- (11 12)><- (21)><- NIL>>> - (4 6 8)>> 
<AbSm - <CrPr - <<GBar<- (21 12)><- (41)><- NIL>>> - (6 6 8)>> 
<AbSm - <CrPr - <<GBar<- (10 23)><- (50)><- NIL>>> - (2 10 8)>> 
<AbSm - <CrPr - <<GBar<- (30 32)><- (30)><- NIL>>> - (2 4 14)>> 
<AbSm - <CrPr - <<GBar<- (20 21)><- (30)><- NIL>>> - (12 15 18)>> 

(? tt 4 gmsm)  ==>
\end{verbatim}}
\newpage
{\footnotesize\begin{verbatim}
----------------------------------------------------------------------{CMBN 3}
<1 * <CrPr - <<GBar<- (10 23)><- (50)><- NIL>>> - (2 10 8)>>
<1 * <CrPr - <<GBar<- (11 12)><- (21)><- NIL>>> - (4 6 8)>>
<1 * <CrPr - <<GBar<- (20 21)><- (30)><- NIL>>> - (12 15 18)>>
<-1 * <CrPr - <<GBar<- (21 12)><- (41)><- NIL>>> - (6 6 8)>>
<-1 * <CrPr - <<GBar<- (30 32)><- (30)><- NIL>>> - (2 4 14)>>
------------------------------------------------------------------------------

(? tt *)  ==>

----------------------------------------------------------------------{CMBN 2}
------------------------------------------------------------------------------
\end{verbatim}}

We may also build the Ronald Brown reduction and test the differential
upon a simplex belonging to the twisted tensor product (slot {\tt :bcc} of the
reduction).
{\footnotesize\begin{verbatim}
(setf br (brown-reduction tw))  ==>

[K59 Reduction]

(setf tw-pr (bcc br))  ==>

[K57 Chain-Complex]

(? tw-pr 4 (tnpr 4 (gbar 4 0 '(1 10 100) 0 '(1000 10000) 0 '(100000) 0 '())
                           0 '()))

----------------------------------------------------------------------{CMBN 3}
<-1 * <TnPr <<GBar>> (111 11000 100000)>>
<1 * <TnPr <<GBar>> (111 101000 10000)>>
<1 * <TnPr <<GBar>> (11011 100 100000)>>
<-1 * <TnPr <<GBar>> (11011 100000 100)>>
<-1 * <TnPr <<GBar>> (101001 110 10000)>>
<1 * <TnPr <<GBar>> (101001 10010 100)>>
<1 * <TnPr <<GBar<- (100000)><- NIL>>> (100)>>
<1 * <TnPr <<GBar<- (1 110)><- (101000)><- NIL>>> NIL>>
<1 * <TnPr <<GBar<- (10 100)><- (10000)><- NIL>>> NIL>>
<-1 * <TnPr <<GBar<- (11 100)><- (11000)><- NIL>>> NIL>>
<1 * <TnPr <<GBar<- (1000 10000)><- (100000)><- NIL>>> NIL>>
<-1 * <TnPr <<GBar<- (1001 10010)><- (100000)><- NIL>>> NIL>>
------------------------------------------------------------------------------

(? tw-pr *)  ==>

----------------------------------------------------------------------{CMBN 2}
------------------------------------------------------------------------------
\end{verbatim}}

\section {The essential contraction for simplicial groups fibrations}

It is known that if ${\cal G}$ is a $0$--reduced simplicial group, 
the space ${\cal WG}=\bar{\cal W}{\cal G} \times_\tau  {\cal G}$ is
contractible. So it is possible to build a reduction of this space over ${\Z}$.
This reduction depends on the important contraction
$$\chi_{\times\tau}:  {\bar{\cal W}{\cal G} \times_\tau  {\cal G}} \longrightarrow 
                      {\bar{\cal W}{\cal G} \times_\tau  {\cal G}}, $$
defined by:
$$\chi_{\times\tau}[(g_{n-1}, \ldots, g_0), g_n] = (-1)^{n+1} (g_n, g_{n-1}, \ldots, g_0, *),$$
where $g_n \in {\cal G}$, $(g_{n-1}, \ldots, g_0)$ is a gbar in $\bar{\cal W}{\cal G}$,  and
$*$ is the neutral element  of ${\cal G}_{n+1}$ (in fact the $n$--th degeneracy of the base point of
${\cal G}$). We recall also that $\chi_{\times\tau}$
is a homotopy operator (degree: $+1$) satisfying the relation
$$d \circ \chi_{\times\tau} + \chi_{\times\tau} \circ d =1.$$
In addition, if we consider the Brown reduction
$$
\diagram{
{\bar{\cal W}{\cal G} \times_\tau  {\cal G}} & \stackrel{h}{\longrightarrow} & 
 {}^s{\bar{\cal W}{\cal G} \times_\tau  {\cal G}} \cr
 {\scriptstyle f} \downarrow \uparrow {\scriptstyle g}  \cr
 {\bar{\cal W}{\cal G} \otimes_\tau  {\cal G}} \cr
}
$$
there is an induced contraction 
$$\chi_{\otimes t}:  {\bar{\cal W}{\cal G} \otimes_\tau  {\cal G}} \longrightarrow 
                     {\bar{\cal W}{\cal G} \otimes_\tau  {\cal G}}, $$
defined by $\chi_{\otimes t}= f \circ \chi_{\times\tau} \circ g$.
This implies that the twisted tensor product is also contractible over ${\Z}$.
\par
The two functions of {\tt Kenzo} for this construction are:
\vskip 0.35cm
{\parindent=0mm
{\leftskip=5mm 
{\tt smgr-crts-contraction} {\em smgr} \hfill {\em [Function]} \par}
{\leftskip=15mm 
Return the homotopy morphism corresponding to the contraction
$\chi_{\times\tau}$. The lisp definition is:
{\footnotesize\begin{verbatim}
                     (defun SMGR-CRTS-CONTRACTION (smgr)
                         (the morphism
                           (build-mrph
                            :sorc (fibration-total (smgr-fibration smgr))
                            :trgt (fibration-total (smgr-fibration smgr))
                            :degr +1
                            :intr (smgr-crts-contraction-intr (bspn smgr))
                            :strt :gnrt
                            :orgn `(smgr-crts-contraction ,smgr))))
\end{verbatim}}
At execution time, the work is essentialy done by the function put in the
slot {\tt :intr}. This function is itself built by the internal function {\tt smgr-crts-contraction-intr}
requiring as argument the base point of the  simplicial group. \par}
{\leftskip=5mm 
{\tt smgr-tnpr-contraction} {\em smgr} \hfill {\em [Function]} \par}
{\leftskip=15mm 
Return the induced morphism $\chi_{\otimes t}= f \circ \chi_{\times\tau} \circ g$,
where $f$ and $g$ are built by the Brown reduction.
The lisp definition is:
{\footnotesize\begin{verbatim}
                     (defun SMGR-TNPR-CONTRACTION (smgr 
                         &aux (fibration (smgr-fibration smgr))
                              (brown (brown-reduction fibration))
                              (f (f brown))
                              (g (g brown))
                              (chi (smgr-crts-contraction smgr)))

                           (the morphism
                              (i-cmps f chi g)))
\end{verbatim}} \par} 
}

\subsection* {Examples}

It is an instructive exercise to build and check the reduction over ${\Z}$ of the space
${\cal WG}=\bar{\cal W}{\cal G} \times_\tau  {\cal G}$ as we did in the
loop spaces fibrations chapter. 
$$
\diagram{
\bar{\cal W}{\cal G} \times_\tau  {\cal G} & \stackrel{\chi_{\times t}}{\longrightarrow} &
{}^s\bar{\cal W}{\cal G} \times_\tau  {\cal G} \cr
 {\scriptstyle aug} \downarrow \uparrow {\scriptstyle coaug}  \cr
 {{\cal C}_*({\Z})} \cr
}
$$
In this case, the two homomorphisms
$f$ and $g$ of the reduction, are respectively the {\em augmentation} and {\em coaugmentation}
morphisms. 
We define both morphisms according to our previous example based on $K({\Z},1)$.
The unit chain complex corresponding to ${\Z}$ is built by the function
{\tt z-chcm}; its unique generator in degree $0$ is {\tt :z-gnrt}. 
{\footnotesize\begin{verbatim}
(setf k1 (k-z-1))  ==>

[K1 Abelian-Simplicial-Group]

(setf tw (smgr-fibration k1))  ==>

[K25 Fibration]

(setf tt (fibration-total tw))  ==>

[K31 Kan-Simplicial-Set]
\end{verbatim}}
The base point of the total space, needed for the coaugmentation, may be obtained
by either of the following statements:
{\footnotesize\begin{verbatim}
(bspn tt)  ==>

<CrPr - <<GBar>> - NIL>

(crpr 0 +null-gbar+ 0 nil)  ==>

<CrPr - <<GBar>> - NIL>

(setf aug (build-mrph
              :sorc (fibration-total(smgr-fibration(k-z-1))) 
              :trgt (z-chcm) 
              :degr 0
              :intr #'(lambda (degr gnrt)
                       (if (zerop degr)
                           (term-cmbn 0 1 :z-gnrt)
                           (zero-cmbn degr)))
              :strt :gnrt 
              :orgn '(aug-fibr-tot-smgr-fibr-k-z-1)  ))  ==>

[K38 Cohomology-Class (degree 0)]

(setf coaug (build-mrph
              :sorc (z-chcm) 
              :trgt (fibration-total(smgr-fibration(k-z-1))) 
              :degr 0
              :intr #'(lambda (degr gnrt)
                         (if (zerop degr)
                             (term-cmbn 0 1 (crpr 0 +null-gbar+ 0 '()))
                             (zero-cmbn degr)))
              :strt :gnrt
              :orgn '(coaug-fibr-tot-smgr-fibr-k-z-1) ))  ==>

[K39 Morphism (degree 0)]

(? aug 0 (bspn tt))  ==>

----------------------------------------------------------------------{CMBN 0}
<1 * Z-GNRT>
------------------------------------------------------------------------------

(? coaug 0 :z-gnrt)  ==>

----------------------------------------------------------------------{CMBN 0}
<1 * <CrPr - <<GBar>> - NIL>>
------------------------------------------------------------------------------

(setf chi-x-tau (smgr-crts-contraction (k-z-1)))  ==>

[K40 Morphism (degree 1)]
\end{verbatim}}
Let us apply the contraction morphism upon some simplices:
{\footnotesize\begin{verbatim}
(setf *tc* (cmbn 0 1 (crpr 0 +null-gbar+ 0 '())))  ==>

----------------------------------------------------------------------{CMBN 0}
<1 * <CrPr - <<GBar>> - NIL>>
------------------------------------------------------------------------------

(? chi-x-tau *)  ==>

----------------------------------------------------------------------{CMBN 1}
------------------------------------------------------------------------------

(setf *tc* (cmbn 4 1 (crpr 0 (gbar 4 0 '(10 11 12) 0 '(20 21) 0 '(30) 0 '())
                           0 '(2 4 6 8))))  ==>

----------------------------------------------------------------------{CMBN 4}
<1 * <CrPr - <<GBar<- (10 11 12)><- (20 21)><- (30)><- NIL>>> - (2 4 6 8)>>
------------------------------------------------------------------------------

(? chi-x-tau *)  ==>

----------------------------------------------------------------------{CMBN 5}
<-1 * <CrPr - <<GBar<- (2 4 6 8)><- (10 11 12)><- (20 21)><- (30)><- NIL>>> 
              4-3-2-1-0 NIL>>
------------------------------------------------------------------------------
\end{verbatim}}
\newpage
{\footnotesize\begin{verbatim}
(setf *tc* (cmbn 3 1 (crpr 0 (gbar 3 0 '(10 11) 0 '(20) 0 '())
                           0 '(2 4 6))))  ==>

----------------------------------------------------------------------{CMBN 3}
<1 * <CrPr - <<GBar<- (10 11)><- (20)><- NIL>>> - (2 4 6)>>
------------------------------------------------------------------------------

(? chi-x-tau *)  ==>

----------------------------------------------------------------------{CMBN 4}
<1 * <CrPr - <<GBar<- (2 4 6)><- (10 11)><- (20)><- NIL>>> 3-2-1-0 NIL>>
------------------------------------------------------------------------------
\end{verbatim}}
We may build now the reduction over ${\Z}$ and test it upon the various simplices
above:
{\footnotesize\begin{verbatim}
(setf rdct (build-rdct :f aug 
                       :g coaug 
                       :h chi-x-tau 
                       :orgn '(reduction-tt-z)))   ==>

[K41 Reduction]

(pre-check-rdct rdct)  ==>

---done---

(setf *tc* (cmbn 0 1 (crpr 0 +null-gbar+ 0 '())))  ==>

----------------------------------------------------------------------{CMBN 0}
<1 * <CrPr - <<GBar>> - NIL>>
------------------------------------------------------------------------------

(setf *bc* (cmbn 0 1 :z-gnrt))  ==>

----------------------------------------------------------------------{CMBN 0}
<1 * Z-GNRT>
------------------------------------------------------------------------------

(check-rdct)  ==>

*TC* => 
----------------------------------------------------------------------{CMBN 0}
<1 * <CrPr - <<GBar>> - NIL>>
------------------------------------------------------------------------------
\end{verbatim}}
\newpage
{\footnotesize\begin{verbatim}
*BC* => 
----------------------------------------------------------------------{CMBN 0}
<1 * Z-GNRT>
------------------------------------------------------------------------------

              .......  All results null ........

---done---

(setf *tc* (cmbn 4 1 (crpr 0 (gbar 4 0 '(10 11 12) 0 '(20 21) 0 '(30) 0 '())
                           0 '(2 4 6 8))))  ==>

----------------------------------------------------------------------{CMBN 4}
<1 * <CrPr - <<GBar<- (10 11 12)><- (20 21)><- (30)><- NIL>>> - (2 4 6 8)>>
------------------------------------------------------------------------------

(check-rdct)  ==>

              .......  All results null ........

---done---

(setf *tc* (cmbn 3 1 (crpr 0 (gbar 3 0 '(10 11) 0 '(20) 0 '())
                             0 '(2 4 6))))  ==>

----------------------------------------------------------------------{CMBN 3}
<1 * <CrPr - <<GBar<- (10 11)><- (20)><- NIL>>> - (2 4 6)>>
------------------------------------------------------------------------------

(check-rdct)  ==>

              .......  All results null ........

---done---
\end{verbatim}}
\vskip 0.30cm
We build now the reduction over ${\Z}$ of the twisted tensor product. The contraction
is obtained by a call to the function {\tt smgr-tnpr-contraction}. 
$$
\diagram{
\bar{\cal W}{\cal G} \otimes_\tau  {\cal G} & \stackrel{\chi_{\otimes t}}{\longrightarrow} &
{}^s\bar{\cal W}{\cal G} \otimes_\tau  {\cal G} \cr
 {\scriptstyle aug} \downarrow \uparrow {\scriptstyle coaug}  \cr
 {{\cal C}_*({\Z})} \cr
}
$$
Of course, we have to define 
the augmentation and coaugmentation morphisms adapted to this new example. The user will note
that the twisted tensor product chain complex may be obtained from the slot {\tt :sorc} of the contration
$\chi_{\otimes t}$.
{\footnotesize\begin{verbatim}
(setf chi-t-tau (smgr-tnpr-contraction (k-z-1)))  ==>

[K100 Morphism (degree 1)]

(setf aug-t (build-mrph
              :sorc (sorc chi-t-tau)
              :trgt (z-chcm) 
              :degr 0
              :intr #'(lambda (degr gnrt)
                       (if (zerop degr)
                           (term-cmbn 0 1 :z-gnrt)
                           (zero-cmbn degr)))
              :strt :gnrt 
              :orgn '(aug-t-fibr-tot-smgr-fibr-k-z-1)  ))   ==>

[K209 Cohomology-Class (degree 0)]

(setf coaug-t (build-mrph
                :sorc (z-chcm) 
                :trgt (sorc chi-t-tau)
                :degr 0
                :intr #'(lambda (degr gnrt)
                         (if (zerop degr)
                             (term-cmbn 0 1 (tnpr 0 +null-gbar+ 0 '()))
                             (zero-cmbn degr)))
                :strt :gnrt
                :orgn '(coaug-t-fibr-tot-smgr-fibr-k-z-1) ))  ==>

[K210 Morphism (degree 0)]

(setf rdct-t (build-rdct :f aug-t 
                         :g coaug-t 
                         :h chi-t-tau
                         :orgn '(reduction-t-tt-z)))   ==>

[K211 Reduction]

(pre-check-rdct rdct-t)

---done---

(setf *tc* (cmbn 0 1 (tnpr 0 +null-gbar+ 0 '())))  ==>

----------------------------------------------------------------------{CMBN 0}
<1 * <TnPr <<GBar>> NIL>>
------------------------------------------------------------------------------

(setf *bc* (cmbn 0 1 :z-gnrt))  ==>

----------------------------------------------------------------------{CMBN 0}
<1 * Z-GNRT>
------------------------------------------------------------------------------

(check-rdct)  ==>

.......  All results null .....

---done---
\end{verbatim}}
The check is validated also for the following simplices. We see also that,
if we apply the contraction $\chi_{\otimes t}$ upon various simplices outside the base fiber, the
result is in general non--null. The nullity of this contraction  outside
the base fiber, which we have verified experimentally in many cases in  loop spaces fibration 
is not verified in  simplicial group fibrations.
{\footnotesize\begin{verbatim}

(setf *tc* (cmbn 3 1 (tnpr 0 +null-gbar+ 3 '(1 10 100))))  ==>

----------------------------------------------------------------------{CMBN 3}
<1 * <TnPr <<GBar>> (1 10 100)>>
------------------------------------------------------------------------------

(? chi-t-tau *tc*)  ==>

----------------------------------------------------------------------{CMBN 4}
<1 * <TnPr <<GBar<- (1 10 100)><1-0 NIL><0 NIL><- NIL>>> NIL>>
------------------------------------------------------------------------------

(setf *tc* (cmbn 3 1 (tnpr 2 (gbar 2 0 '(1) 0 '()) 1 '(10))))  ==>

----------------------------------------------------------------------{CMBN 3}
<1 * <TnPr <<GBar<- (1)><- NIL>>> (10)>>
------------------------------------------------------------------------------

(? chi-t-tau *tc*)  ==>

----------------------------------------------------------------------{CMBN 4}
<1 * <TnPr <<GBar<1-0 (10)><1-0 NIL><- (1)><- NIL>>> NIL>>
<-1 * <TnPr <<GBar<2-0 (10)><1 (1)><0 NIL><- NIL>>> NIL>>
<1 * <TnPr <<GBar<2-1 (10)><0 (1)><0 NIL><- NIL>>> NIL>>
------------------------------------------------------------------------------

(setf *tc* (cmbn 3 1 (tnpr 3 (gbar 3 0 '(1 10) 0 '(100) 0 '()) 0 '())))  ==>

----------------------------------------------------------------------{CMBN 3}
<1 * <TnPr <<GBar<- (1 10)><- (100)><- NIL>>> NIL>>
------------------------------------------------------------------------------

(? chi-t-tau *tc*)  ==>

----------------------------------------------------------------------{CMBN 4}
<1 * <TnPr <<GBar<2 (1 10)><1 (100)><0 NIL><- NIL>>> NIL>>
<-1 * <TnPr <<GBar<2-1 (11)><0 (100)><0 NIL><- NIL>>> NIL>>
------------------------------------------------------------------------------

(setf *tc* (cmbn 4 1 (tnpr 0 +null-gbar+ 4 '(1 10 100 1000))))  ==>

----------------------------------------------------------------------{CMBN 4}
<1 * <TnPr <<GBar>> (1 10 100 1000)>>
------------------------------------------------------------------------------

(? chi-t-tau *tc*)  ==>

----------------------------------------------------------------------{CMBN 5}
<-1 * <TnPr <<GBar<- (1 10 100 1000)><2-1-0 NIL><1-0 NIL><0 NIL><- NIL>>> NIL>>
------------------------------------------------------------------------------

(setf *tc* (cmbn 4 1 (tnpr 2 (gbar 2 0 '(1) 0 '()) 2 '(10 100))))  ==>

----------------------------------------------------------------------{CMBN 4}
<1 * <TnPr <<GBar<- (1)><- NIL>>> (10 100)>>
------------------------------------------------------------------------------

(? chi-t-tau *tc*)  ==>

----------------------------------------------------------------------{CMBN 5}
<-1 * <TnPr <<GBar<1-0 (10 100)><2-1-0 NIL><1-0 NIL><- (1)><- NIL>>> NIL>>
<1 * <TnPr <<GBar<2-0 (10 100)><2-1-0 NIL><1 (1)><0 NIL><- NIL>>> NIL>>
<-1 * <TnPr <<GBar<2-1 (10 100)><2-1-0 NIL><0 (1)><0 NIL><- NIL>>> NIL>>
<-1 * <TnPr <<GBar<3-0 (10 100)><2-1 (1)><1-0 NIL><0 NIL><- NIL>>> NIL>>
<1 * <TnPr <<GBar<3-1 (10 100)><2-0 (1)><1-0 NIL><0 NIL><- NIL>>> NIL>>
<-1 * <TnPr <<GBar<3-2 (10 100)><1-0 (1)><1-0 NIL><0 NIL><- NIL>>> NIL>>
------------------------------------------------------------------------------

(setf *tc* (cmbn 4 1 (tnpr 3 (gbar 3 0 '(1 10) 0 '(100) 0 '())
                           1 '(1000))))   ==>
\end{verbatim}}
\newpage
{\footnotesize\begin{verbatim}
----------------------------------------------------------------------{CMBN 4}
<1 * <TnPr <<GBar<- (1 10)><- (100)><- NIL>>> (1000)>>
------------------------------------------------------------------------------

(? chi-t-tau *tc*)  ==>

----------------------------------------------------------------------{CMBN 5}
<-1 * <TnPr <<GBar<2 (1 10 1000)><2-1-0 NIL><1 (100)><0 NIL><- NIL>>> NIL>>
<1 * <TnPr <<GBar<2-1 (11 1000)><2-1-0 NIL><0 (100)><0 NIL><- NIL>>> NIL>>
<-1 * <TnPr <<GBar<2-1-0 (1000)><2-1-0 NIL><- (1 10)><- (100)><- NIL>>> NIL>>
<1 * <TnPr <<GBar<3 (1 10 1000)><2-1 (100)><1-0 NIL><0 NIL><- NIL>>> NIL>>
<-1 * <TnPr <<GBar<3 (1 1000 10)><2-1 (100)><1-0 NIL><0 NIL><- NIL>>> NIL>>
<1 * <TnPr <<GBar<3 (1000 1 10)><2-0 (100)><1-0 NIL><0 NIL><- NIL>>> NIL>>
<-1 * <TnPr <<GBar<3-1 (11 1000)><2-0 (100)><1-0 NIL><0 NIL><- NIL>>> NIL>>
<1 * <TnPr <<GBar<3-1-0 (1000)><2 (1 10)><1-0 NIL><- (100)><- NIL>>> NIL>>
<1 * <TnPr <<GBar<3-2 (11 1000)><1-0 (100)><1-0 NIL><0 NIL><- NIL>>> NIL>>
<-1 * <TnPr <<GBar<3-2 (1000 11)><1-0 (100)><1-0 NIL><0 NIL><- NIL>>> NIL>>
<-1 * <TnPr <<GBar<3-2-0 (1000)><1 (1 10)><1 (100)><0 NIL><- NIL>>> NIL>>
<1 * <TnPr <<GBar<3-2-1 (1000)><0 (1 10)><0 (100)><0 NIL><- NIL>>> NIL>>
------------------------------------------------------------------------------

(setf *tc* (cmbn 4 1 (tnpr 4 (gbar 4 0 '(1 10 100)
                                     0 '(1000 10000)
                                     0 '(100000)
                                     0 '())
                              0 '())))     ==>

----------------------------------------------------------------------{CMBN 4}
<1 * <TnPr <<GBar<- (1 10 100)><- (1000 10000)><- (100000)><- NIL>>> NIL>>
------------------------------------------------------------------------------

(? chi-t-tau *tc*)  ==>

----------------------------------------------------------------------{CMBN 5}
<1 * <TnPr <<GBar<2 (1001 10010 100)><2-1-0 NIL><1 (100000)><0 NIL><- NIL>>> NIL>>
<-1 * <TnPr <<GBar<2-1 (11011 100)><2-1-0 NIL><0 (100000)><0 NIL><- NIL>>> NIL>>
<-1 * <TnPr <<GBar<3 (1 10 100)><2 (1000 10000)><1-0 NIL><- (100000)><- NIL>>> NIL>>
<-1 * <TnPr <<GBar<3 (111 1000 10000)><2-0 (100000)><1-0 NIL><0 NIL><- NIL>>> NIL>>
<1 * <TnPr <<GBar<3 (1001 110 10000)><2-1 (100000)><1-0 NIL><0 NIL><- NIL>>> NIL>>
<-1 * <TnPr <<GBar<3 (1001 10010 100)><2-1 (100000)><1-0 NIL><0 NIL><- NIL>>> NIL>>
<1 * <TnPr <<GBar<3-1 (11011 100)><2-0 (100000)><1-0 NIL><0 NIL><- NIL>>> NIL>>
<1 * <TnPr <<GBar<3-2 (1 110)><1 (1000 10000)><1 (100000)><0 NIL><- NIL>>> NIL>>
<1 * <TnPr <<GBar<3-2 (111 11000)><1-0 (100000)><1-0 NIL><0 NIL><- NIL>>> NIL>>
<-1 * <TnPr <<GBar<3-2 (11011 100)><1-0 (100000)><1-0 NIL><0 NIL><- NIL>>> NIL>>
<-1 * <TnPr <<GBar<3-2-1 (111)><0 (1000 10000)><0 (100000)><0 NIL><- NIL>>> NIL>>
------------------------------------------------------------------------------
\end{verbatim}}
\newpage
Let us verify once again the unproved conjecture signaled in the loop spaces fibrations
chapter. In fact, if the conjecture is true, it is sufficient to prove
it with $\bar{\Delta}$. But the following is also a good test of verifying all the
involved machinery. If we take ${\cal G}= K({\Z},1)$, the space 
${\cal WG}=\bar{\cal W}{\cal G} \times_\tau  {\cal G}$ is $0$--reduced. Let us consider
its loop space and the induced contraction $\chi_{\otimes t}$ over
${\cal WG} \otimes_\tau \Omega{\cal WG}$. ${\cal WG}$ and the contraction $\chi_{\otimes t}$
are  built by the two following statements:
{\footnotesize\begin{verbatim}
(setf wgkz1 (fibration-total (smgr-fibration(k-z-1))))  ==>

[K31 Kan-Simplicial-Set]

(setf chi-t-tau (tnpr-contraction wgkz1))  ==>

[K84 Morphism (degree 1)]
\end{verbatim}}
Let us apply successively the contraction upon  a simplex belonging to
the base fiber ({\tt gnrt0}) and a simplex out of the base fiber ({\tt gnrt1}).
We see that in the second case we obtain a null result. We verify also that
$\chi_{\otimes t}\circ \chi_{\otimes t}=0$.
{\footnotesize\begin{verbatim}
(setf gnrt0 (tnpr 0 (crpr 0 +null-gbar+  0 nil)
                  0 (loop3 0 (crpr 0 (gbar 2 0 '(1) 0 '(0) ) 0 '(1)) -3)))  ==>

<TnPr <CrPr - <<GBar>> - NIL> <<Loop[<CrPr - <<GBar<- (1)><- (0)>>> - (1)>\-3]>>>

(? chi-t-tau 0 gnrt0)  ==>

----------------------------------------------------------------------{CMBN 1}
<1 * <TnPr <CrPr - <<GBar<- (1)><- (0)>>> - (1)> 
           <<Loop[<CrPr - <<GBar<- (1)><- (0)>>> - (1)>\-3]>>>>
<1 * <TnPr <CrPr - <<GBar<- (1)><- (0)>>> - (1)> 
           <<Loop[<CrPr - <<GBar<- (1)><- (0)>>> - (1)>\-2]>>>>
<1 * <TnPr <CrPr - <<GBar<- (1)><- (0)>>> - (1)> 
           <<Loop[<CrPr - <<GBar<- (1)><- (0)>>> - (1)>\-1]>>>>
------------------------------------------------------------------------------

(? chi-t-tau *)  ==>

----------------------------------------------------------------------{CMBN 2}
------------------------------------------------------------------------------

(setf gnrt1 (tnpr 4 (crpr 0 (gbar 4 0 '(10 11 12) 0 '(20 21) 0 '(30) 0 '())
                          0 '(2 4 6 8))
                  0 (loop3 0 (crpr 0 (gbar 2 0 '(1) 0 '(0) ) 0 '(1)) 2)))  ==>

<TnPr <CrPr - <<GBar<- (10 11 12)><- (20 21)><- (30)><- NIL>>> - (2 4 6 8)> 
      <<Loop[<CrPr - <<GBar<- (1)><- (0)>>> - (1)>\2]>>>
\end{verbatim}}
\newpage
{\footnotesize\begin{verbatim}
(? chi-t-tau 4 gnrt1)  ==>

----------------------------------------------------------------------{CMBN 5}
------------------------------------------------------------------------------
\end{verbatim}}
At last we verify that the total space of the fibration of $K({\Z},1)$ is
contractible.
{\footnotesize\begin{verbatim}
(homology (fibration-total(smgr-fibration (k-z-1))) 0 10)  ==>

Homology in dimension 0 :

Component Z

Homology in dimension 1 :

Homology in dimension 2 :

Homology in dimension 3 :

Homology in dimension 4 :

Homology in dimension 5 :

Homology in dimension 6 :

Homology in dimension 7 :

Homology in dimension 8 :

Homology in dimension 9 :

---done---
\end{verbatim}}
Another example is the following:
we build the canonical fibration of $\bar{\cal W}^2K({\Z_2}, 1)$ and verify 
that its fibration total  space is contractible:
{\footnotesize\begin{verbatim}
(setf k22 (classifying-space(classifying-space(k-z2-1))))  ==>

[K306 Abelian-Simplicial-Group]

(setf fb (smgr-fibration k22))  ==>

[K836 Fibration]

(setf tt2 (fibration-total fb))  ==>

[K837 Kan-Simplicial-Set]

(homology tt2 0 10)  ==>

Homology in dimension 0 :

Component Z

Homology in dimension 1 :

Homology in dimension 2 :

Homology in dimension 3 :

Homology in dimension 4 :

Homology in dimension 5 :

Homology in dimension 6 :

Homology in dimension 7 :

Homology in dimension 8 :

Homology in dimension 9 :

--done--
\end{verbatim}}

\subsection* {Lisp file concerned in this chapter}

{\tt cs-twisted-products.lisp}.


