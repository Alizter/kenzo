\chapter {Computing homotopy groups}

\section {Mathematical aspects}

The\index{computing homotopy groups} method used here to compute the homotopy groups is  known as the {\bf Whitehead tower}.
We recall some mathematical points:

\subsection* {1) The Hurewicz theorem.}

{\bf Theorem.}
{\em Let X be a $1$--connected space such that all its homology groups $H_r(X)$ are null
for $0<r<n$, then $\pi_n(X)\simeq H_n(X)$.}

\subsection* {2) The properties of the classifying spaces.}

Let $X$ such that $\pi_r(X)=0$ for $0<r<n$. From now on, we shall denote  $\pi_n(X)$ simply by $\pi$.
Let us consider the universal coefficients exact sequence:
$$ 0 \longrightarrow Ext(H_{n-1}(X), \pi) \longrightarrow H^n(X,\pi) \longrightarrow Hom(H_n(X),\pi) \longrightarrow 0.$$
Here $H_{n-1}(X)=0$, so that a canonical isomorphism is given:
$$H^n(X,\pi) \buildrel {\simeq} \over \longrightarrow Hom(H_n(X),\pi),$$
but $H_n(X) \simeq \pi$ and this gives us a canonical element $h \in H^n(X, \pi)$
which is the fundamental $n$--th cohomology class in this context. A cocycle $\chi$ representing $h$ is constructed as
follows. In the chain complex
$$ \cdots  \rightarrow  C_{n+1}  \buildrel {d_{n+1}} \over \longrightarrow C_n
            \buildrel {d_n} \over \longrightarrow  C_{n-1} \longrightarrow \cdots,$$
the kernel $Z_n$ of $d_n$ is a summand of $C_n(X)$:
$$ C_n = Z_n \oplus S$$
and such a decomposition induces a projection $p:C_n(X) \longrightarrow Z_n$.
Finally, we have also the canonical projection $p':Z_n \longrightarrow H_n=\pi$.
Then, $\chi$ can be defined by $\chi = p' \circ p$ and $\chi$ {\em is a cocycle}. If another
$p$ is chosen, another $\chi$ is obtained but the cohomology class
is the same.
\par
Let us fix now a choice for the cocycle $\chi: C_n(X) \longrightarrow \pi$. This cocycle induces in turn
a simplicial map:
$$ \varphi_\chi: C_n(X) \longrightarrow K(\pi,n), $$
where $K(\pi,n)$ is the classifying group $K(\pi_n(X),n)$ of the group $\pi_n(X)=\pi$.
If $\chi'$ is another choice for the cocycle then $\varphi_{\chi'}$ is homotopic to $\varphi_\chi$.
We recall that the space $K(\pi,n)$ has the following properties:
\begin{enumerate}
\item $H_r(K(\pi,n))=0$ for $0 < r < n$,
\item $\pi_r(K(\pi,n))=0$ for $r \not= n$,
\item $\pi_n(K(\pi,n))=H_n(K(\pi,n)= \pi$.
\end{enumerate}
On the other hand, a canonical fibration of base space $K(\pi, n)$ and fibre space
$K(\pi, n-1)$ is deduced from a canonical twisting operator $$\tau: K(\pi, n) \longrightarrow K(\pi, n-1)$$
$$\diagram{
K(\pi, n-1) \cr
\downarrow \cr
K(\pi, n) \times_\tau K(\pi, n-1) \cr
\downarrow \cr
K(\pi,n) }
$$

So, if a simplicial map $\varphi_\chi: X \longrightarrow K(\pi,n)$ is given,
it is then possible to construct a fibration of base space $X$ and fibre space $K(\pi, n-1)$,
the twisting operator being then defined by: $\tau_\chi = \tau \circ \varphi_\chi $, according to the diagram:
$$\diagram{
K(\pi, n-1) & \buildrel {=} \over \longrightarrow & K(\pi, n-1) \cr
\downarrow  & & \downarrow \cr
X'=X \times_{\tau_\chi} K(\pi, n-1) & \buildrel {\psi_\chi} \over \longrightarrow & K(\pi,n) \times_\tau K(\pi, n-1) \cr
\downarrow  & & \downarrow \cr
   X        & \buildrel \varphi_\chi \over \longrightarrow & K(\pi,n) }
$$

\subsection* {3) The Serre exact sequence of the homotopy groups of fibrations.}

Let us consider the fibration
$$  F \quad \hookrightarrow\quad  T=X \otimes_\tau F \hskip 1.2em \longrightarrow \hskip -1.2em \longrightarrow  
   \quad X,$$
The Serre exact sequence establishes a connection between the homotopy groups of the $3$ spaces,
in any  dimension:
$$\cdots \longrightarrow \pi_{n+1}(F) \longrightarrow \pi_{n+1}(T) \longrightarrow \pi_{n+1}(X) 
         \longrightarrow \pi_n(F) \longrightarrow$$
$$\longrightarrow  \pi_n(T) \longrightarrow \pi_n(X)  \longrightarrow \pi_{n-1}(F) \longrightarrow \cdots $$
In our special case, where the fibration is:
$$  K(\pi, n-1) \quad \hookrightarrow\quad  X' \hskip 1.2em \longrightarrow \hskip -1.2em \longrightarrow  
   \quad X,$$
the Serre sequence may be written:
$$\cdots \longrightarrow  \pi_{n+1}(K(\pi, n-1)) \longrightarrow \pi_{n+1}(X') \longrightarrow \pi_{n+1}(X)
 \longrightarrow $$
$$ \longrightarrow \pi_n(K(\pi, n-1)) \longrightarrow  \pi_n(X') \longrightarrow \pi_n(X) 
 \longrightarrow \pi_{n-1}(K(\pi, n-1)) \longrightarrow 0. $$
But we know that $\pi_i(K(\pi,n))=0$ for $i \not= n$ and that $\pi_n(K(\pi, n))=\pi,$
so the exact sequence may be re-written:
$$  0 \rightarrow \pi_{n+1}(X') \rightarrow \pi_{n+1}(X) \rightarrow 0 \rightarrow  \pi_n(X') 
      \rightarrow \pi_n(X) \rightarrow \pi \rightarrow 0. $$
The  subsequence:
$$ 0 \longrightarrow  \pi_n(X') \longrightarrow \pi_n(X) 
 \longrightarrow \pi \longrightarrow 0. $$
is exact; furthermore, the cocycle $\chi$ is such that the connection morphism
$\partial: \pi_n(X) \rightarrow \pi$ is the identity; so that
we deduce that $ \pi_n (X')=0 $.
\par
On the other hand, from the exactness of the subsequence
$$  0 \longrightarrow \pi_{n+1}(X') \longrightarrow \pi_{n+1}(X)
 \longrightarrow 0 $$
we deduce that $ \pi_{n+1}(X') \simeq \pi_{n+1}(X)$
and more generally $ \pi_r(X') \simeq \pi_r(X)$ for $r \not= n.$
In particular, $\pi_r(X')=0$ for $r \leq n+1$ and the Hurewicz theorem gives again:
$$ H_{n+1}(X') \simeq \pi_{n+1}(X') \simeq  \pi_{n+1}(X).$$
So, if we know how to compute $H_{n+1}(X')$ then we  have obtained $\pi_{n+1}(X).$

\subsection* {4) The Whitehead tower}

Let\index{Whitehead tower} us denote for a reason which will be clear in a moment the space $X'$ by
$X^{(n+1)}$.
Due to the properties of  $X^{(n+1)}$, we may iterate the process, 
namely, build the following fibrations, where $\pi_{n+1}(X^{(n+1)})$ is  denoted simply by
$\pi'$:
$$\diagram{
K(\pi', n) &\buildrel {=} \over \longrightarrow  & K(\pi', n) \cr
\downarrow  & & \downarrow \cr
X^{(n+2)}=X^{(n+1)} \times_{\tau_\chi} K(\pi', n) & \buildrel {\psi_\chi} \over \longrightarrow & 
K(\pi',n+1) \times_\tau K(\pi', n) \cr
\downarrow  & & \downarrow \cr
   X^{(n+1)}  & \buildrel \varphi_\chi \over \longrightarrow & K(\pi',n+1) }
$$
where $X^{(n+2)}= X^{(n+1)} \times_{\tau_\chi} K(\pi', n)$ has the property
$$H_{n+2}(X^{(n+2)}) \simeq  \pi_{n+2}(X^{(n+2)}) \simeq \pi_{n+2}(X^{(n+1)}) \simeq \pi_{n+2}(X). $$
This construction is known as the Whitehead tower. 

\section {The functions for computing homotopy groups}

\subsection* {An important remark}

In this version of Kenzo, only the case where the first non--null
homology group (in non--null dimension) is ${\Z}$ or ${\Z}/{2 {\Z}}$ can be processed; however if this 
homology group is a direct sum of several copies of ${\Z}$ or ${\Z}/{2 {\Z}}$, then the corresponding
stage of the Whitehead tower may also be constructed step by step.
\vskip 0.35cm
{\parindent=0mm
{\leftskip=5mm
{\tt chml-clss} {\em chcm first}  \hfill {\em [Function]} \par}
{\leftskip=15mm
Return a ``fundamental'' cohomology class, more precisely a cocycle $\chi$ defining it.
The first argument {\em chcm} must be a chain complex $C_*$ with effective homology; in particular, 
an effective chain complex $EC_*$ is a by--product of the machine object $C_*$. The second argument ``{\em first}''
is an integer $n$, namely the first non--null dimension from which the chain complex
{\em chcm} has a non--null homology group, which {\bf must be isomorphic to ${\Z}$} or ${\Z}/{2 {\Z}}$.
The reader may be amazed that this argument ``{\em first}'' must be provided, since it is  a consequence
of the given $C_*$. But in fact, the same function {\tt chml-class} may also be used in different contexts, for example,
the Postnikov tower; in such a case, the argument {\em first} is not redundant. The returned cocycle $\chi$
is in any case a chain complex morphism $\chi: EC_* \longrightarrow {\Z}$, where ${\Z}$ is the unit chain complex
created by the function {\tt z-chcm}. The degree of the morphism is $- n$.
It is important to note that  the chain complex involved in the source of the morphism
is the {\bf effective} chain complex of the homotopy equivalence value of the slot {\tt efhm} of
the object {\em chcm}. See the section {\em The general method for computing homology} in the Homology chapter.
Finally, if $H_n(C_*)={\Z}/{2 {\Z}}$, the actual cohomology class hoped by the user is the composition
$p \circ \chi$, where $p$ is the canonical projection ${\Z} \longrightarrow {\Z}/{2 {\Z}}$. But nevertheless,
the {\tt chml-clss} lisp function returns $\chi$ and not $p \circ \chi$.
\par}
{\leftskip=5mm
{\tt z-whitehead} {\em smst chml-clss}  \hfill {\em [Function]} \par}
{\leftskip=15mm
Return a fibration  over the simplicial set {\em smst} (the first argument), canonically associated
to the ``cohomology class'' {\em chml-clss} (the second argument). The simplicial set $X$, i.e. {\em smst}, is 
reduced; its homotopy groups $\pi_r(X)$ are null for $0 \leq r \leq n-1$. The first non null homotopy group
$\pi_n(X)$ is assumed to be ${\Z}$, i.e. $\pi_n(X)= H_n(X)= {\Z}$. The previous function {\tt chml-clss}, in
this situation, returns a cocycle $\chi$, which may be used as the second argument {\em chml-clss} of the function
{\tt z-whitehead}. The integer  $n$ is determined by the absolute value of the degree of the
cohomology class {\em chcm-clss}. As explained in the previous section, a canonical fibration
is induced by $\chi$ and it is this fibration which is returned by {\tt z-whitehead}. The slot {\tt sintr} of the
fibration is set by the internal function {\tt z-whitehead-sintr} which builds 
in an efficient way the lisp function implementing the twisting operator $\tau \circ \varphi_\chi$.
\par}
{\leftskip=5mm
{\tt z2-whitehead} {\em smst chcm-clss}  \hfill {\em [Function]} \par}
{\leftskip=15mm
Return a fibration  over the simplicial set {\em smst} (the first argument), canonically associated
to the ``cohomology class'' {\em chml-clss} (the second argument). The simplicial set $X$, i.e. {\em smst}, is 
reduced; its homotopy groups $\pi_r(X)$ are null for $0 \leq r \leq n-1$. The first non null homotopy group
$\pi_n(X)$ is assumed to be ${\Z}/2{\Z}$, i.e. $\pi_n(X)= H_n(X)= {\Z}/2{\Z}$. The previous function {\tt chml-clss}, in
this situation, returns a ``cocycle'' $\chi$, which may be used as the second argument {\em chml-clss} of the function
{\tt z2-whitehead}. The integer $n$ is determined by the absolute value of the degree of the
cohomology class {\em chcm-clss}. In this ${\Z}/2{\Z}$ case, $\chi$ is even not a cocycle. The actual cocycle is
obtained by the composition $p \circ \chi$, $p$ being the canonical projection 
${\Z} \longrightarrow {\Z}/{2 {\Z}}$. But the user is not concerned by these technicalities, because the function
{\tt z2-whitehead} makes itself the necessary conversion. The slot {\tt sintr} of the
fibration is set by the internal function {\tt z2-whitehead-sintr} which builds 
in an efficient way the lisp function implementing the twisting operator $\tau \circ \varphi_\chi$.
\par}
}

\subsection* {Examples}

Let us retrieve some known facts about $S^3$, in particular $\pi_4(S^3) \simeq {\Z}/{2 {\Z}}$. 
We follow the theoritical method exposed above, namely  build the fibration
$$  K({\Z},2) \quad \hookrightarrow\quad  S^3 \times_{\tau_\chi} K({\Z},2) \hskip 1.2em 
    \longrightarrow \hskip -1.2em \longrightarrow     \quad S^3,$$
and get the homology group in dimension $4$ of the total space.
{\footnotesize\begin{verbatim}
(setf s3 (sphere 3))  ==>

[K1 Simplicial-Set]

(homology s3 0 4)  ==>

Homology in dimension 0 :

Component Z

Homology in dimension 1 :

Homology in dimension 2 :

Homology in dimension 3 :

Component Z

---done---

(setf s3-chml-clss (chml-clss s3 3))  ==>

[K12 Cohomology-Class (degree 3)]

(setf s3-fibr (z-whitehead s3 s3-chml-clss))  ==>

[K37 Fibration]

(setf s3-total (fibration-total  s3-fibr))  ==>

[K43 Simplicial-Set]

(homology s3-total 0 6)

Homology in dimension 0 :

Component Z

Homology in dimension 1 :

Homology in dimension 2 :

Homology in dimension 3 :

Homology in dimension 4 :

Component Z/2Z

Homology in dimension 5 :

---done---
\end{verbatim}}
Let us show a similar example with the space $Moore(2,4)$. As $H_4 \simeq {\Z}/{2 {\Z}}$, the function
for building the fibration is {\tt z2-whitehead}. 
We verify that $\pi_5(Moore(2,4))\simeq {\Z}/{2 {\Z}}$
{\footnotesize\begin{verbatim}
(setf m24 (moore 2 4)) ==>

[K1 Simplicial-Set]

(show-structure m24 6)  ==>

Dimension = 0 :

        Vertices :  (*)

Dimension = 1 :

Dimension = 2 :

Dimension = 3 :

Dimension = 4 :

        Simplex : M4

                Faces : (<AbSm 2-1-0 *> <AbSm 2-1-0 *> <AbSm 2-1-0 *> 
                         <AbSm 2-1-0 *> <AbSm 2-1-0 *>)
Dimension = 5 :

        Simplex : N5

                Faces : (<AbSm - M4> <AbSm 3-2-1-0 *> <AbSm - M4> <AbSm 3-2-1-0 *> 
                         <AbSm 3-2-1-0 *> <AbSm 3-2-1-0 *>)

Dimension = 6 :

NIL

(homology m24 0 5)  ==>

Homology in dimension 0 :

Component Z

Homology in dimension 1 :

Homology in dimension 2 :

Homology in dimension 3 :

Homology in dimension 4 :

Component Z/2Z

---done---

(setf m24-chml-clss (chml-clss m24 4))  ==>

[K12 Cohomology-Class (degree 4)]

(setf m24-fibr (z2-whitehead m24 m24-chml-clss))  ==>

[K49 Fibration]

(setf m24-total (fibration-total m24-fibr))  ==>

[K55 Simplicial-Set]

(homology m24-total 0 6)  ==>

Homology in dimension 0 :

Component Z

Homology in dimension 1 :

Homology in dimension 2 :

Homology in dimension 3 :

Homology in dimension 4 :

Homology in dimension 5 :

Component Z/2Z

---done---
\end{verbatim}}

We may even verify that, up to $9$, the homotopy groups of the classifying group $K({\Z},5)$, for instance,
are null except $\pi_5$. This is a far-fetched method to verify this well known result, but
it proves that the software is coherent.
{\footnotesize\begin{verbatim}
(setf k5 (k-z 5))  ==>

[K49 Abelian-Simplicial-Group]

(setf k5-chml-clss (chml-clss k5 5))  ==>

[K576 Cohomology-Class (degree 5)]

(setf k5-fibr (z-whitehead k5 k5-chml-clss))  ==>

[K579 Fibration]

(setf k5-total (fibration-total k5-fibr))  ==>

[K580 Kan-Simplicial-Set]

(homology k5-total 0 10)  ==>

Homology in dimension 0 :

Component Z

Homology in dimension 1 :

Homology in dimension 2 :

Homology in dimension 3 :

Homology in dimension 4 :

Homology in dimension 5 :

Homology in dimension 6 :

Homology in dimension 7 :

Homology in dimension 8 :

Homology in dimension 9 :

---done---
\end{verbatim}}

Let us show now the iteration of the process, to get for instance
$\pi_5(S^3)$ and $\pi_6(S^3)$:
{\footnotesize\begin{verbatim}
(setf s3 (sphere 3))  ==>

[K1 Simplicial-Set]

(setf s3-chml-clss (chml-clss s3 3))  ==>

[K12 Cohomology-Class (degree 3)]

(setf s3-fibration (z-whitehead s3 s3-chml-clss))  ==>

[K37 Fibration]

(setf s3-4 (fibration-total s3-fibration))  ==>

[K43 Simplicial-Set]

(homology s3-4 4)  ==>

Homology in dimension 4 :

Component Z/2Z

---done---

(setf s3-4-chml-clss (chml-clss s3-4 4))  ==>

[K253 Cohomology-Class (degree 4)]

(setf s3-4-fibration (z2-whitehead s3-4 s3-4-chml-clss))  ==>

[K292 Fibration]

(setf s3-5 (fibration-total s3-4-fibration))  ==>

[K298 Simplicial-Set]

(homology s3-5 5)  ==>

Homology in dimension 5 :

Component Z/2Z

---done---

(setf s3-5-chml-clss (chml-clss s3-5 5))  ==>

[K609 Cohomology-Class (degree 5)]

(setf s3-5-fibration (z2-whitehead s3-5 s3-5-chml-clss))  ==>

[K624 Fibration]

(setf s3-6 (fibration-total s3-5-fibration))  ==>

[K630 Simplicial-Set]

(homology s3-6 6)  ==>

Component Z/12Z
\end{verbatim}}

An interesting example is given by the real projective space $P^\infty{\R}/P^1{\R}$ which may be built in
{\tt Kenzo} by the function {\tt r-proj-space}. We list its first homotopy groups.
{\footnotesize\begin{verbatim}
(setf x (r-proj-space 2))  ==>

[K1 Simplicial-Set]

(homology x 0 10)  ==>

Homology in dimension 0 :

Component Z

Homology in dimension 1 :

Homology in dimension 2 :

Component Z

Homology in dimension 3 :

Component Z/2Z

Homology in dimension 4 :

Homology in dimension 5 :

Component Z/2Z

Homology in dimension 6 :

Homology in dimension 7 :

Component Z/2Z

Homology in dimension 8 :

Homology in dimension 9 :

Component Z/2Z

---done---
\end{verbatim}}
Using the machinery described in this chapter, we find that the total space of the fibration
has the same homology groups as $S^3$. It has been effectively proved that this space is homotopic
to $S^3$. 

{\footnotesize\begin{verbatim}
(setf ch (chml-clss x 2))  ==>

[K12 Cohomology-Class (degree 2)]

(setf f2 (z-whitehead x ch))  ==>

[K25 Fibration]

(setf x3 (fibration-total f2))  ==>

[K31 Simplicial-Set]

(homology x3 0 10)  ==>

Homology in dimension 0 :

Component Z

Homology in dimension 1 :

Homology in dimension 2 :

Homology in dimension 3 :

Component Z

Homology in dimension 4 :

Homology in dimension 5 :

Homology in dimension 6 :

Homology in dimension 7 :

---done---
\end{verbatim}}

\subsection* {Lisp files concerned in this chapter}

{\tt whitehead.lisp}, {\tt smith.lisp}.
