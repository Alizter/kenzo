\chapter {Simplicial morphisms}

The software {\tt Kenzo} implements simplicial morphisms in a way analogous to
chain complex morphisms.\index{simplicial morphisms}

\section {Representation of a  simplicial morphism}

A simplicial morphism is  an instance\index{class!{\tt SIMPLICIAL-MRPH}} of the class {\tt SIMPLICIAL-MRPH}, 
subclass of the class {\tt MORPHISM}.
{\footnotesize\begin{verbatim}
(DEFCLASS SIMPLICIAL-MRPH (morphism)
  ((sintr :type sintr :initarg :sintr :reader sintr)))
\end{verbatim}}
It has one slot of its own:
\begin{description}
\item {{\tt sintr}}, an object of type {\tt SINTR}, in fact a lisp function defining the morphism
between the source and target simplicial sets. It must have  $2$ parameters:  a dimension (an integer)
and a geometric simplex of this dimension (a generator of any type) .
It must return  an abstract simplex, image in the target simplicial set of this geometric simplex.
\end{description}
A printing method has been associated to the class {\tt SIMPLICIAL-MRPH}.
A string like {\tt [K{\em n} Simplicial-Morphism]} is the  external representation of an instance of
such a class,  where {\em n} is the number plate of the {\tt Kenzo} object.

\section {The function build-smmr}

To\index{function!{\tt build-smmr}} facilitate the construction of instances 
of the class {\tt SIMPLICIAL-MRPH} and to free  the user to call
the standard constructor {\tt make-instance}, the software provides the function
\vskip 0.35cm
{\tt build-smmr :sorc} {\em sorc} {\tt :trgt} {\em trgt} {\tt :degr} {\em degr} {\tt :sintr} {\em sintr}
{\tt :intr} {\em intr} \par
\hspace*{22.5mm}{\tt :strt} {\em strt} {\tt :orgn} {\em orgn}
\vskip 0.35cm
defined with keyword parameters and returning an instance of the class {\tt SIMPLICIAL-MRPH}.
The keyword arguments of {\tt build-smmr} are:

\begin{itemize}
\item [--] {\em sorc}, the source object, an object of type {\tt SIMPLICIAL-SET}.
\item [--] {\em trgt}, the target object, an object of type {\tt SIMPLICIAL-SET}.
\item [--] {\em degr}, the degree of the morphism, an integer. In this chapter, we  consider only the
$0$ degree case (the usual one). The case $-1$ is particularly important: it allows to implement the notion
of twisting operator defining a fibration (See the chapter Fibration).
\item [--] {\em sintr}, the internal lisp function defining the effective mapping between simplicial sets.
If the integer {\em degr} is $0$ and if the following keyword argument {\em intr} is omitted, then 
the function {\tt build-smmr} builds a lisp function implementing the induced mapping between the
underlying source and target chain complexes. This function is installed in the slot {\tt intr}.
The strategy is then set to {\tt :gnrt}.
\item [--] {\em intr}, a user defined morphism for the underlying chain complexes. This argument is optional and
taken in account only if the degree is $0$ and in this case, supersedes the previous derived  mapping. 
The strategy is then mandatory.
If the degree is not null, the implementor has decided  to set the corresponding slot to {\tt NIL}.
\item [--] {\em strt}, the strategy, i.e. {\tt :gnrt} or {\tt :cmbn} attached to the previous function.
\item [--] {\em orgn}, a relevant comment list.
\end{itemize}

After a call to {\tt build-smmr}, the simplicial morphism instance 
is added to a list of previously constructed  ones ({\tt *smmr-list*}). 
As the other similar lists,
the list {\tt *smmr-list*} may be cleared by the function {\tt cat-init}.
\par
The effective application of a simplicial morphism upon arguments, 
is re\-a\-li\-zed with the macro {\tt ?} which  calls the adequate method
defined for this kind of objects.
\newpage
{\parindent=0mm
{\leftskip=5mm
{\tt ? \&rest} {\em args} \hfill {\em [Macro]} \par}
{\leftskip=15mm
Versatile macro for applying a simplicial morphism  indifferently
as {\tt (? {\em smmr dmns absm-or-gmsm})} or  {\tt (? {\em smmr cmbn})}. In the first case, the
third argument is either an abstract simplex or a geometric one. In the second case, if the
second argument is a combination, {\em smmr} is then considered as a chain complex  morphism and
it is the function in the slot {\tt intr} which is applied, so that it makes sense
only if the degree of the mapping is $0$. \par}
}

\subsection* {Examples}

In the following examples, we work with $\Delta_3$. We define three simplicial morphisms, {\tt sm1}, {\tt sm2}
and {\tt sm3}. In {\tt sm1}, the mapping is the identity mapping and we can see that this identity mapping
has been propagated on the underlying chain complex.
{\footnotesize\begin{verbatim}
(setf d3 (delta 3))  ==>

[K1 Simplicial-Set]

(setf sm1 (build-smmr
            :sorc d3 :trgt d3 :degr 0
            :sintr #'(lambda (dmns gmsm)
                        (absm 0 gmsm))
            :orgn '(identity delta-3)))   ==>

[K6 Simplicial-Morphism]

(? sm1 2 7)  ==>

<AbSm - 7>

(? sm1 1 (absm 1 1))  ==>

<AbSm 0 1>

(? sm1 (cmbn 2 1 7))  ==>

----------------------------------------------------------------------{CMBN 2}
<1 * 7>
------------------------------------------------------------------------------
\end{verbatim}}
In the simplicial morphism {\tt sm2}, the mapping is the ``null'' mapping. In
fact, it consists in  applying any abstract simplex in dimension $n$ upon the $n$--degenerate base point. 
Of course, in terms of chain complex, the cor\-res\-pon\-ding mapping applies any chain complex element
upon the null combination of same degree.
{\footnotesize\begin{verbatim}
(setf sm2 (build-smmr
            :sorc d3 :trgt d3 :degr 0
            :sintr #'(lambda (dmns gmsm)
                       (absm (mask dmns) 1))    ;;; mask(n)=2^n - 1 
            :orgn '(null delta-3)))  ==>

[K7 Simplicial-Morphism]

(? sm2 0 4)  ==>

<AbSm - 1>

(? sm2 2 7)  ==>

<AbSm 1-0 1>

(? sm2 1 (absm 1 1))  ==>

<AbSm 0 1>

(? sm2 3 15)  ==>

<AbSm 2-1-0 1>

(? sm2 (cmbn 3 2 15))  ==>

----------------------------------------------------------------------{CMBN 3}
------------------------------------------------------------------------------
\end{verbatim}}
In the simplicial morphism {\tt sm3}, we keep the same ``null'' mapping as in {\tt sm2}
but we choose as mapping for the underlying chain complex the opposite of any chain complex 
element.
{\footnotesize\begin{verbatim}
(setf sm3 (build-smmr
            :sorc d3 :trgt d3 :degr 0
            :sintr #'(lambda (dmns gmsm)
                       (absm (mask dmns) 1))
            :intr #'cmbn-opps
            :strt :cmbn
            :orgn '(null and opposite delta-3)))  ==>

[K8 Simplicial-Morphism]

(? sm3 3 15)  ==>

<AbSm 2-1-0 1>

(? sm3 (cmbn 3 7 15))  ==>

----------------------------------------------------------------------{CMBN 3}
<-7 * 15>
------------------------------------------------------------------------------

(? sm3 (cmbn 3 11 (dlop-ext-int '(0 1 2 3))))  ==>

----------------------------------------------------------------------{CMBN 3}
<-11 * 15>
------------------------------------------------------------------------------
\end{verbatim}}

\subsection* {Lisp files concerned in this chapter}

{\tt simplicial-mrphs.lisp}.
\par
[{\tt classes.lisp}, {\tt macros.lisp}, {\tt various.lisp}].
